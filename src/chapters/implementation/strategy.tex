\section{Strategy Implementation}
\label{sec:impl-strategies}

This section presents the practical realization of the four architectural strategies introduced in Section~\ref{sec:architectural-strategies}. While the conceptual motivations were discussed earlier, the focus here is on their concrete execution: preparation of transcripts and few-shot examples, invocation of language models, and consolidation of outputs. Simplified TypeScript sketches of the core procedures are included in Appendix~\ref{app:strategy-impl}.

\subsection{S1 Implementation: Single-Pass Full-Input}
\label{subsec:impl-s1}

The S1 strategy is implemented through the \texttt{singleLlmAllField} function, which processes the entire template in a single model call. The system first constructs a combined transcript and retrieves a small set of examples from the RAG agent. All fields are represented within a unified schema, and the prompt incorporates task instructions, field definitions, transcripts, current values, and retrieved examples.

A single \texttt{generateObject} call produces a complete candidate template. The output is subsequently normalized via the CF agent’s rules and verified once in aggregate. The strategy is therefore characterized by a single prompt, a single model invocation, and batch-style post-processing. While efficient and simple to orchestrate, S1 provides limited error isolation, as malformed responses may affect all fields simultaneously.

\subsection{S2 Implementation: Iterative Single-Field}
\label{subsec:impl-s2}

S2, implemented by \texttt{singleLlmOneField}, introduces field-level decomposition. After preparing the combined transcript and retrieving few-shot examples, the system constructs individual prompts for each field, containing only the information relevant to that field, including filtered examples.

Each field is processed by an independent asynchronous task. Outputs pass through normalization and are merged into the final template. Errors are isolated to the corresponding field; upon failure, the system reverts to a sensible fallback (typically the previously known value). S2 thus provides improved robustness relative to S1 through parallelized extraction, field-specific prompting, and localized error handling.

\subsection{S3 Implementation: Multi-LLM Consensus}
\label{subsec:impl-s3}

The S3 strategy is implemented using \texttt{dualLlmAllField}, which performs model-level parallelism. A unified prompt and schema are constructed as in S1, but two model providers (e.g., GPT-4 and Gemini~2.0 Flash) are invoked in parallel. Because both models receive identical prompts, differences in their outputs reflect model behaviour, not prompt variation.

The two candidates are forwarded to an ensemble verifier, which applies rule-based checks and a judge LLM to select, for each field, the more reliable value or retain the original value when neither candidate is adequate. A finalized template and confidence metadata are produced. S3 is therefore defined by parallel model inference, an explicit consensus phase, and structured documentation of ensemble decisions.

\subsection{S4 Implementation: Multi-LLM Per-Field}
\label{subsec:impl-s4}

S4, implemented through \texttt{multiLlmOneField}, combines field-wise decomposition with multi-model consensus. For each field, the system constructs a tailored prompt and invokes both model providers. Their outputs are passed to a per-field verifier, which applies the same decision logic as in S3 but at field granularity.

Fields are processed sequentially to avoid rate-limit interactions and to simplify auditing. Each iteration returns a \texttt{FilledField} entry after consensus and normalization. This yields the highest robustness and interpretability among the strategies, though at the cost of substantially increased computational overhead.

\subsection{Strategy Orchestration Infrastructure}
\label{subsec:impl-orchestration-strategies}

All strategies are executed through a dedicated \texttt{StrategyOrchestrator}, which exposes a uniform entry point (\texttt{executeStrategy}). The orchestrator maps the selected identifier (S1--S4) to a concrete implementation adhering to a shared \texttt{StrategyImplementation} interface. 

Common infrastructure ensures consistent error handling, telemetry, and configuration. Strategy-specific parameters (model names, temperatures, provider settings) are injected through environment-based configuration, and all outputs are serialized into a unified \texttt{FinalTemplate} structure, ensuring that downstream evaluation and analytics remain strategy-agnostic. A minimal TypeScript sketch of the orchestrator is provided in Appendix~\ref{app:strategy-impl}.
