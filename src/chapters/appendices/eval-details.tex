\section{Curated Test Suite Overview}
\label{app:dataset-overview}

The curated evaluation suite maintained in \texttt{code-guardian/evaluation/datasets/} contains representative vulnerability snippets for JavaScript/TypeScript secure coding. Expected vulnerabilities include CWE identifiers and severities so results can be aggregated by category.

\subsection*{Dataset sizes}

At the time of writing, the suite contains:
\begin{itemize}
  \item \textbf{Core dataset:} 20 test cases (18 vulnerable, 2 secure)
  \item \textbf{Advanced dataset:} 28 test cases (25 vulnerable, 3 secure)
\end{itemize}
The repository-contained evaluation harness loads the core dataset by default; the advanced dataset is included to broaden coverage and can be evaluated by extending the harness.

\subsection*{Representative Vulnerability Classes}

The dataset includes (non-exhaustive) examples for:
\begin{itemize}
  \item SQL injection (CWE-89)
  \item Cross-site scripting (CWE-79)
  \item Command injection (CWE-78)
  \item Path traversal (CWE-22)
  \item Insecure randomness (CWE-330)
  \item Hardcoded credentials (CWE-798)
  \item Insecure CORS / CSRF patterns (CWE-352 and related)
\end{itemize}

\subsection*{Test Case Record Format}

Each test case contains:
\begin{itemize}
  \item a code snippet (\texttt{code}),
  \item a list of expected findings (\texttt{expectedVulnerabilities}),
  \item and optional remediation guidance (\texttt{expectedFix}).
\end{itemize}

\subsection*{Reproducing the harness run}

The evaluation script is executed locally:
\begin{lstlisting}[language=Java, caption={Running the evaluation script}, label={lst:appendix-eval-run}]
cd code-guardian
node evaluation/evaluate-models.js
\end{lstlisting}

The script prints per-model precision/recall/F1, false positive rate, average response time, and JSON parse success rate. Models to test are specified in the script and can be edited to match the locally installed Ollama models.

This appendix is intentionally concise: the full dataset is machine-readable and can be inspected directly in the repository.
