This chapter analyzes the problem space of automated vulnerability detection and repair within modern software development workflows. The objective is to identify the fundamental technical and practical requirements for integrating Large Language Models (LLMs) into secure coding assistance, to critically examine existing approaches, and to derive design implications for a privacy-preserving, retrieval-augmented framework integrated into an Integrated Development Environment (IDE).


\section{Requirements}
\label{sec:requirements}

This section outlines the core requirements that a system must fulfill to support effective, privacy-preserving vulnerability detection and repair within modern software development workflows. The focus of this thesis is on assisting developers during implementation by identifying security-relevant weaknesses in source code and providing actionable repair suggestions directly within the Integrated Development Environment (IDE).

The primary goal of the system proposed in this thesis is to enable developers to detect and remediate vulnerabilities in JavaScript and TypeScript code using a locally deployed Large Language Model (LLM) augmented with retrieval-based security knowledge. Rather than relying on cloud-hosted services or post hoc pipeline analysis, the system operates entirely on the developer’s machine and integrates seamlessly into Visual Studio Code. Source code, analysis results, and vulnerability knowledge remain local at all times, ensuring privacy, reproducibility, and suitability for regulated or security-sensitive environments.

In contrast to traditional Static Application Security Testing (SAST) tools, which rely on predefined rules and often provide limited remediation guidance, the proposed system leverages LLM-based semantic reasoning combined with Retrieval-Augmented Generation (RAG). This enables the system to reason about code context, explain detected issues, and suggest concrete repairs grounded in curated vulnerability knowledge. To ensure practical usefulness, the system must satisfy both functional and non-functional requirements related to accuracy, transparency, performance, and usability.

Based on the challenges identified in Chapter~\ref{chap:introduction} and the analysis of existing approaches, six core requirements (R1–R6) are defined. These requirements establish a systematic basis for evaluating the proposed architecture and guide the design decisions discussed in subsequent chapters.

\textbf{R1 – Detection Accuracy and Consistency:}  
The system must reliably identify security-relevant vulnerabilities in source code with stable behavior across repeated analyses. Detection results should be consistent for identical inputs, independent of invocation timing or interaction mode. The system should minimize false positives while maintaining adequate recall, ensuring that developers can trust reported findings and are not overwhelmed by spurious warnings.

\textbf{R2 – Context-Aware Vulnerability Reasoning:}  
The system must analyze vulnerabilities in their surrounding code context rather than relying solely on syntactic patterns. This includes reasoning about data flow, API usage, and control structures at the function and file level. The system should correctly distinguish between vulnerable and benign code patterns that appear syntactically similar and must avoid flagging issues that are mitigated by contextual safeguards.

\textbf{R3 – Explainability and Transparency:}  
To foster developer trust and facilitate efficient remediation, the system must provide transparent explanations for detected vulnerabilities. Each finding should be accompanied by a clear description of the underlying weakness, references to relevant vulnerability classes (e.g., CWE), and an indication of which code fragments contributed to the detection. Explanations should be concise, technically precise, and suitable for developers without specialized security expertise.

\textbf{R4 – Actionable Repair Suggestions:}  
The system must generate concrete, security-aware repair suggestions that address the root cause of detected vulnerabilities while preserving functional correctness. Suggested fixes should be directly applicable to the affected code region and must avoid introducing new security issues or breaking existing behavior. Developers must retain full control over whether and how suggested changes are applied.

\textbf{R5 – Privacy-Preserving Operation:}  
All analysis, retrieval, and generation steps must be performed locally without transmitting source code or derived artifacts to external services. The system must operate with locally deployed models and locally stored vulnerability knowledge, ensuring that proprietary code remains confidential and that results are reproducible across environments.

\textbf{R6 – Usability and Responsiveness:}  
The system must integrate smoothly into the IDE and provide feedback within acceptable latency bounds. Inline detection should support near–real-time interaction to avoid disrupting the development flow, while more comprehensive audit operations may tolerate higher latency. Usability is evaluated with respect to end-to-end response time, clarity of presented information, and compatibility with typical developer hardware configurations.

These six requirements define the evaluation criteria for the proposed system. In the following chapters, each requirement is addressed through specific architectural choices and implementation strategies. The evaluation chapter assesses to what extent the system satisfies these requirements in practice, using quantitative metrics and reproducible benchmarks.

\subsection{R1: Detection Accuracy and Consistency}\vspace{-2mm}

\label{sec:r1-consistency}

Consistency refers to the system’s ability to produce stable, repeatable, and uniformly structured vulnerability detection results when analyzing identical or semantically equivalent source code. In the context of security analysis, consistency means that the same vulnerability is detected, classified, explained, and localized in a comparable manner across repeated runs, invocation modes, and interaction contexts.

Unlike general-purpose code analysis, vulnerability detection demands a high degree of determinism. Security findings are often used to guide remediation decisions, trigger audits, or satisfy compliance requirements. Inconsistent detection outcomes—such as reporting a vulnerability in one analysis but not in another, or fluctuating between different vulnerability classes—undermine developer trust and reduce the practical usability of the system. More generally, LLM-based systems are known to exhibit non-deterministic behavior and hallucination under unconstrained generation, motivating strict prompting and grounding strategies for stable outputs \cite{openai2023gpt4,ji2023hallucination,pearce2022copilot}.

% As illustrated in Figure~\ref{fig:ide-detection-example}, an
An ideal solution should ensure that once a vulnerability is identified, it is reported in a consistent manner. This includes stable classification (e.g., mapping to the same CWE category), consistent localization of the affected code region, and uniform explanation structure. For example, an input validation flaw should not alternately be reported as a generic "security issue," an injection vulnerability, or a logic error across different executions if the underlying code has not changed.

Consistency operates across multiple dimensions of vulnerability reporting. At the level of detection outcome, the presence or absence of a vulnerability should be stable across repeated analyses. At the level of classification, detected issues should map consistently to the same vulnerability categories and severity levels. At the level of explanation, descriptions should follow standardized phrasing and structure, avoiding unnecessary variation in terminology or level of detail. At the level of localization, the same code regions should be highlighted as relevant to the vulnerability.

This requirement is particularly critical because modern development workflows increasingly rely on automated security feedback integrated into the IDE. If a vulnerability warning appears intermittently or changes classification without code modifications, developers may disregard the warning entirely. Similar concerns have been documented in studies of static analysis tools, where inconsistent or noisy warnings reduce adoption and remediation rates \cite{johnson2013don}.

From a system design perspective, achieving consistency requires controlling sources of nondeterminism in LLM inference and grounding vulnerability reasoning in structured security knowledge. Retrieval-Augmented Generation contributes to this goal by anchoring model outputs to curated vulnerability descriptions and examples, thereby reducing reliance on purely generative reasoning and mitigating hallucination.

For evaluation, detection consistency is assessed using repeated analyses of identical code samples under fixed configurations. We measure agreement between runs using macro-averaged precision, recall, and F1-score for vulnerability presence and classification. In addition, label agreement metrics are used to quantify stability in vulnerability categorization across runs. A high degree of agreement indicates that the system produces stable and reliable security findings.

\begin{table}[h!]
\centering
\renewcommand{\arraystretch}{1.6}
\setlength{\tabcolsep}{12pt}
\begin{tabularx}{\textwidth}{|>{\centering\arraybackslash}m{3cm}|>{\arraybackslash}X|}
\hline
\textbf{Consistency Level} & \textbf{Interpretation (example thresholds)} \\
\hline
High &
\textbf{High consistency.} Vulnerability presence and classification are stable across runs (macro F1 $\geq 0.80$), with minimal variation in localization and explanation structure. \\
\hline
Medium &
\textbf{Moderate consistency.} Minor variations in classification or explanation occur (macro F1 in $[0.65, 0.80)$), but core vulnerability detection remains stable. \\
\hline
Low &
\textbf{Low consistency.} Frequent changes in vulnerability presence or classification (macro F1 in $[0.50, 0.65)$), indicating unstable detection behavior. \\
\hline
None &
\textbf{No consistency.} Detection results vary substantially across runs (macro F1 $< 0.50$), undermining trust in the system. \\
\hline
\end{tabularx}
\caption{Evaluation scale for R1: Detection Consistency (example thresholds).}
\label{tab:r1-detection-consistency}
\end{table}

In summary, R1 ensures that vulnerability detection results are stable, reproducible, and uniformly structured across repeated analyses. By enforcing consistency across detection outcomes, classification, explanation, and localization, the system provides developers with reliable security feedback that can be trusted and acted upon within real-world development workflows.
 

\subsection{R2: Context-Aware Vulnerability Reasoning}\vspace{-2mm}
\label{sec:r2-context-awareness}

Context-aware vulnerability reasoning refers to the system's ability to correctly identify, classify, and localize security vulnerabilities by analyzing source code within its surrounding semantic and structural context. This requirement goes beyond surface-level pattern matching and instead relies on understanding data flow, control flow, API semantics, and usage constraints to determine whether a code fragment constitutes a genuine security risk.

In contrast to traditional static analyzers that operate primarily on syntactic rules or predefined patterns, effective vulnerability reasoning must consider how code behaves in context. Prior research has shown that many vulnerabilities only manifest under specific execution paths, input assumptions, or API usage scenarios, and cannot be reliably detected without contextual analysis \cite{livshits2005java,chess2004staticanalysis}. Similarly, LLM-based approaches that lack explicit grounding may misclassify benign code as vulnerable or overlook subtle security flaws when context is incomplete or fragmented \cite{pearce2022copilot,ji2023hallucination}.

% As illustrated in Figure~\ref{fig:ide-detection-example}, an
An ideal solution must detect vulnerabilities even when relevant information is distributed across multiple statements, functions, or files. The system should associate related code fragments into a coherent reasoning context while ignoring unrelated logic. For example, input validation performed in a helper function should be correctly recognized when assessing the safety of downstream API usage, and defensive checks should prevent false positives when they effectively mitigate a potential vulnerability.

This requirement is critical because real-world codebases are rarely self-contained or linear. Security-relevant information is often scattered across variable initializations, conditional branches, utility functions, and framework abstractions. Without robust contextual reasoning, a system may either miss vulnerabilities that emerge from interdependent logic or incorrectly flag code that is secure by design. Such errors reduce developer confidence and limit the system’s usefulness in practice.

Effective context-aware reasoning operates along three complementary dimensions. First, the system must correctly integrate dispersed information. Security-relevant signals may appear in different parts of a file or across multiple files, and the system must combine these fragments into a unified vulnerability assessment. Second, the system must recognize mitigation logic. If appropriate safeguards—such as input sanitization, authentication checks, or bounds validation—are present, the system should account for them and avoid reporting false positives. Third, the system must avoid unsupported inference. When insufficient context is available to determine whether a vulnerability exists, the system should explicitly acknowledge uncertainty rather than hallucinating a definitive conclusion.

Retrieval-Augmented Generation supports this requirement by grounding vulnerability reasoning in structured security knowledge, such as Common Weakness Enumeration (CWE) descriptions, secure coding guidelines, and historical vulnerability examples. Retrieved context helps the model align observed code patterns with known vulnerability semantics, reducing reliance on implicit assumptions and improving reasoning reliability.

The advantages of strong context-aware vulnerability reasoning are multifold. It improves detection accuracy by reducing false positives and false negatives caused by superficial pattern matching. It enhances robustness to coding style variation by focusing on semantic behavior rather than syntactic form. It also improves developer trust, as reported vulnerabilities more closely align with actual security risks in the codebase.

For evaluation, context-aware reasoning is assessed using classification metrics that measure the correctness of vulnerability detection and categorization in context-rich scenarios. Precision, recall, and macro-averaged F1-score are computed over benchmarks containing both vulnerable and non-vulnerable code samples with similar surface patterns. In addition, localization accuracy is measured by evaluating whether the system correctly identifies the relevant code regions contributing to the vulnerability. These metrics collectively capture the system’s ability to reason about vulnerabilities in context rather than in isolation.

\begin{table}[h!]
\centering
\renewcommand{\arraystretch}{1.6}
\setlength{\tabcolsep}{12pt}
\begin{tabularx}{\textwidth}{|>{\centering\arraybackslash}m{3cm}|>{\arraybackslash}X|}
\hline
\textbf{Reasoning Level} & \textbf{Interpretation (example criteria)} \\
\hline
High &
\textbf{High context awareness.} The system correctly integrates dispersed context, recognizes mitigation logic, and avoids unsupported inference. Vulnerability classification and localization are accurate (macro F1 $\geq 0.80$). \\
\hline
Medium &
\textbf{Moderate context awareness.} The system integrates some contextual information but occasionally misses mitigations or misinterprets dependencies (macro F1 in $[0.65, 0.80)$). \\
\hline
Low &
\textbf{Low context awareness.} The system relies primarily on surface patterns, leading to frequent false positives or missed vulnerabilities (macro F1 in $[0.50, 0.65)$). \\
\hline
None &
\textbf{No context awareness.} The system fails to incorporate contextual information and produces unreliable vulnerability assessments (macro F1 $< 0.50$). \\
\hline
\end{tabularx}
\caption{Evaluation scale for R2: Context-Aware Vulnerability Reasoning.}
\label{tab:r2-context-awareness}
\end{table}

In summary, R2 ensures that vulnerability detection is grounded in semantic and structural understanding of source code rather than superficial pattern matching. By integrating dispersed context, accounting for mitigation logic, and avoiding unsupported assumptions, the system provides accurate and trustworthy vulnerability assessments that reflect real-world security risks in modern codebases.


\subsection{R3: Explainability and Transparency}\vspace{-2mm}
\label{sec:r3-transparency}

Explainability and transparency refer to the system's ability to make its vulnerability detection and repair reasoning understandable, inspectable, and verifiable by developers. In the context of security analysis, transparency means that the system does not merely report that a vulnerability exists, but clearly communicates \emph{why} it was detected, \emph{which code elements contributed to the decision}, and \emph{what security principles are being violated}. This requirement is essential for fostering developer trust, enabling informed remediation decisions, and supporting auditability in security-sensitive environments.

Unlike traditional static analyzers, which often expose explicit rules or taint paths, LLM-based systems risk operating as opaque black boxes. Prior research has shown that when developers cannot understand the rationale behind automated security findings, they are more likely to ignore warnings or apply fixes incorrectly \cite{johnson2013don, christakis2016developers}. This issue is amplified for LLM-based approaches, where probabilistic reasoning and generative explanations may obscure the causal relationship between code patterns and reported vulnerabilities.

% As illustrated in Figure~\ref{fig:ide-detection-example}, an
An ideal solution should present vulnerability findings alongside structured explanations that link detected issues to concrete code regions and recognized vulnerability classes. For example, when reporting an injection vulnerability, the system should indicate the untrusted input source, the absence or insufficiency of validation or sanitization, and the sensitive sink where exploitation may occur. Explanations should be concise, technically precise, and aligned with established security taxonomies such as the Common Weakness Enumeration (CWE).

Explainability operates across several dimensions. At the level of localization, the system should highlight the specific lines or code fragments that contributed to the detection, enabling developers to quickly identify the relevant context. At the level of reasoning, the system should describe the logical chain that led from observed code patterns to the vulnerability conclusion, avoiding vague or purely descriptive statements. At the level of justification, the system should reference recognized vulnerability categories or security guidelines to ground its explanations in established knowledge rather than ad hoc model intuition.

This requirement is particularly important in real-world development workflows, where developers must often balance security concerns against functional requirements and delivery timelines. Transparent explanations allow developers to assess whether a reported issue is relevant in their specific context and to determine whether a suggested fix aligns with project constraints. In regulated domains, transparency further supports accountability by enabling security findings to be reviewed, documented, and justified during audits.

Retrieval-Augmented Generation plays a central role in supporting explainability. By grounding explanations in retrieved vulnerability descriptions, secure coding guidelines, and historical examples, the system can produce explanations that are both informative and consistent. This reduces the risk of hallucinated or misleading justifications and improves alignment between detection outcomes and established security knowledge.

For evaluation, explainability and transparency are assessed through a combination of qualitative and quantitative criteria. We evaluate whether explanations correctly reference the underlying vulnerability type, accurately identify the contributing code regions, and maintain internal coherence between detection, explanation, and suggested repair. In addition, explanation completeness is assessed by verifying that all essential components of the vulnerability reasoning—such as source, sink, and missing mitigation—are explicitly addressed. While explainability is inherently qualitative, structured scoring rubrics enable reproducible assessment across benchmarks.

\begin{table}[h!]
\centering
\renewcommand{\arraystretch}{1.6}
\setlength{\tabcolsep}{12pt}
\begin{tabularx}{\textwidth}{|>{\centering\arraybackslash}m{3cm}|>{\arraybackslash}X|}
\hline
\textbf{Transparency Level} & \textbf{Interpretation (example criteria)} \\
\hline
High &
\textbf{High transparency.} Explanations clearly identify the vulnerability type, affected code regions, and reasoning steps, and reference established security knowledge. Developers can easily verify and act on the findings. \\
\hline
Medium &
\textbf{Moderate transparency.} Explanations identify the vulnerability and affected code but provide limited reasoning detail or incomplete justification. Additional developer interpretation is required. \\
\hline
Low &
\textbf{Low transparency.} Explanations are vague, generic, or loosely connected to the reported vulnerability, making verification difficult. \\
\hline
None &
\textbf{No transparency.} The system reports vulnerabilities without meaningful explanation or justification, effectively operating as a black box. \\
\hline
\end{tabularx}
\caption{Evaluation scale for R3: Explainability and Transparency.}
\label{tab:r3-transparency}
\end{table}

R3 ensures that findings are understandable and reviewable, not just correct in aggregate metrics. Clear links between code evidence, vulnerability class, and suggested action are necessary for developer trust and accountable use.


\subsection{R4: Actionable Repair Suggestions}\vspace{-2mm}
\label{sec:r4-repair-quality}

Actionable repair suggestions refer to the system's ability to generate concrete, security-aware code modifications that effectively remediate detected vulnerabilities while preserving the original program's functional correctness. In contrast to generic advice or high-level recommendations, actionable repairs must be directly applicable to the affected code region and sufficiently specific to support immediate developer adoption.

In the context of vulnerability detection, identifying a security flaw is only the first step. Developers ultimately require guidance on how to fix the issue correctly and efficiently. Empirical evidence suggests that warning overload and unclear remediation guidance reduce adoption and follow-through, especially when developers must interpret findings and design fixes under time pressure \cite{johnson2013don,christakis2016developers}. Automated program repair research also highlights that patch quality and evaluation are non-trivial: fixes must address the root cause without introducing regressions or unintended behavior changes \cite{weimer2009genprog,kim2013par,monperrus2014critique}. Inconsistent or incorrect fixes may leave vulnerabilities partially unresolved or introduce new flaws, undermining the value of automated detection.

% As illustrated in Figure~\ref{fig:ide-repair-example}, an
An ideal solution should provide repair suggestions that are tightly coupled to the detected vulnerability and localized to the relevant code region. For example, if an injection vulnerability is identified, the system should suggest concrete input validation or parameterization mechanisms that are appropriate for the specific API and execution context, rather than issuing abstract recommendations such as "sanitize input." Suggested repairs should reflect established secure coding practices and align with recognized vulnerability classes, such as those defined by the Common Weakness Enumeration (CWE) and practitioner guidance such as OWASP cheat sheets \cite{mitreCWE,owaspCheatSheets,owaspSqlCheatSheet,owaspXssCheatSheet}.

More broadly, actionable repair guidance should cover a range of common weakness classes beyond injection, including CSRF defenses, robust input validation, safe deserialization, secure file upload handling, password storage, and security logging practices \cite{owaspCsrfCheatSheet,owaspInputValidationCheatSheet,owaspDeserializationCheatSheet,owaspFileUploadCheatSheet,owaspPasswordStorageCheatSheet,owaspLoggingCheatSheet}.

Actionable repair suggestions operate across several dimensions. At the level of specificity, suggested fixes must include precise code changes rather than vague guidance. At the level of correctness, repairs must eliminate the underlying vulnerability without breaking existing functionality or introducing new security issues. At the level of contextual appropriateness, fixes should respect the surrounding code structure, library usage, and project conventions, avoiding disruptive or unrealistic refactorings. Finally, at the level of control, developers must retain full authority over whether and how suggested repairs are applied.

This requirement is particularly important in IDE-integrated workflows, where developers expect rapid, low-friction assistance. Repair suggestions that are overly verbose, difficult to interpret, or incompatible with the existing codebase are likely to be ignored. Conversely, concise and correct fixes that can be reviewed and applied incrementally encourage adoption and improve remediation rates.

Retrieval-Augmented Generation supports actionable repair suggestions by grounding generated fixes in curated vulnerability knowledge and historical remediation examples. Retrieved context enables the system to align suggested repairs with established security practices and reduce the risk of hallucinated or insecure fixes. By decoupling security knowledge from the model parameters, RAG further allows repair logic to evolve as new vulnerability patterns and recommended mitigations emerge.

For evaluation, repair quality is assessed using a combination of functional and security-oriented metrics. Functional correctness is evaluated by verifying that repaired code preserves expected behavior, for example through regression tests or benchmark-provided test cases. Security effectiveness is evaluated by re-analyzing the repaired code to confirm that the original vulnerability is no longer detected and that no new vulnerabilities are introduced. In addition, repair precision is assessed by measuring the proportion of suggested fixes that are both applicable and correct without manual modification.

\begin{table}[h!]
\centering
\renewcommand{\arraystretch}{1.6}
\setlength{\tabcolsep}{12pt}
\begin{tabularx}{\textwidth}{|>{\centering\arraybackslash}m{3cm}|>{\arraybackslash}X|}
\hline
\textbf{Repair Quality Level} & \textbf{Interpretation (example criteria)} \\
\hline
High &
\textbf{High-quality repairs.} Suggested fixes are directly applicable, remove the vulnerability, preserve functional correctness, and align with secure coding practices. \\
\hline
Medium &
\textbf{Moderate-quality repairs.} Suggested fixes address the vulnerability but require minor manual adjustment or introduce small, non-critical side effects. \\
\hline
Low &
\textbf{Low-quality repairs.} Suggested fixes are vague, incomplete, or partially incorrect, requiring substantial developer intervention. \\
\hline
None &
\textbf{No actionable repair.} The system fails to provide a usable fix or produces insecure or functionally incorrect code. \\
\hline
\end{tabularx}
\caption{Evaluation scale for R4: Actionable Repair Suggestions.}
\label{tab:r4-repair-quality}
\end{table}

R4 bridges detection and remediation. In practice, suggestions are valuable only when they are specific, context-appropriate, and safe to review incrementally, while final control remains with the developer.


\subsection{R5: Privacy-Preserving Operation}\vspace{-2mm}
\label{sec:r5-privacy}

Privacy-preserving operation refers to the system's ability to perform vulnerability detection, reasoning, and repair generation without exposing source code or derived artifacts to external services. This requirement ensures that all stages of analysis—including model inference, retrieval of security knowledge, and generation of explanations or fixes—are executed locally within the developer's environment.

Source code frequently contains proprietary logic, intellectual property, or sensitive business information. In many industrial, governmental, and regulated settings, transmitting such data to cloud-hosted services is unacceptable due to confidentiality, compliance, or contractual constraints. Consequently, any practical vulnerability detection system intended for real-world adoption must provide strong guarantees that code remains under the developer's control at all times.

% As illustrated in Figure~\ref{fig:local-architecture-overview}, an
An ideal solution should operate entirely on local hardware, using locally deployed LLMs and locally stored vulnerability knowledge bases. No source code, intermediate representations, embeddings, or analysis results should be transmitted beyond the local machine. This includes not only raw code but also prompts, retrieved documents, and generated outputs, all of which may inadvertently leak sensitive information if handled improperly.

Privacy-preserving operation encompasses several dimensions. At the level of deployment, the system must rely exclusively on local inference engines and avoid dependencies on external APIs or remote model hosting. At the level of data handling, all inputs and outputs must remain confined to local memory or storage, with no background telemetry or logging that could result in unintended data egress. At the level of reproducibility, local execution ensures that analysis results can be replicated across environments without reliance on changing external services or opaque model updates.

This requirement is particularly important for vulnerability detection, where analysis often requires access to complete source files or project-level context. Partial redaction or anonymization strategies are insufficient, as they may remove security-relevant information and degrade detection accuracy. By contrast, local execution allows full-context analysis while maintaining strict confidentiality.

Retrieval-Augmented Generation supports privacy-preserving operation by decoupling security knowledge from the model parameters and enabling the use of locally maintained knowledge bases. Vulnerability descriptions, secure coding guidelines, and historical examples can be curated and updated locally without requiring cloud-based retrieval or retraining. This design enables timely incorporation of new vulnerability knowledge while preserving data sovereignty.

For evaluation, privacy preservation is assessed through architectural inspection and runtime verification. We verify that no network communication occurs during analysis by monitoring outbound connections and ensuring that all model inference and retrieval operations are confined to local processes. In addition, we assess whether the system functions correctly in offline environments, confirming that detection and repair capabilities do not degrade when network access is unavailable. These checks provide objective evidence that privacy guarantees are upheld in practice.

\begin{table}[h!]
\centering
\renewcommand{\arraystretch}{1.6}
\setlength{\tabcolsep}{12pt}
\begin{tabularx}{\textwidth}{|>{\centering\arraybackslash}m{3cm}|>{\arraybackslash}X|}
\hline
\textbf{Privacy Level} & \textbf{Interpretation (example criteria)} \\
\hline
High &
\textbf{Strong privacy guarantees.} All analysis stages run locally, no network communication is observed, and the system functions fully offline without loss of capability. \\
\hline
Medium &
\textbf{Partial privacy.} Core analysis is local, but auxiliary components (e.g., optional updates or logging) require network access. No source code is transmitted. \\
\hline
Low &
\textbf{Weak privacy.} Some analysis steps or prompts rely on external services, introducing potential data exposure risks. \\
\hline
None &
\textbf{No privacy guarantees.} Source code or derived artifacts are transmitted to remote services during analysis. \\
\hline
\end{tabularx}
\caption{Evaluation scale for R5: Privacy-Preserving Operation.}
\label{tab:r5-privacy}
\end{table}

In summary, R5 ensures that vulnerability detection and repair can be performed without compromising source code confidentiality. By enforcing fully local execution, eliminating external dependencies, and enabling offline operation, the system addresses a key barrier to adoption of LLM-based security assistance in real-world, security-sensitive development environments.


\subsection{R6: Usability and Responsiveness}\vspace{-2mm}
\label{sec:r6-usability}

Usability in the context of the proposed framework refers primarily to \textbf{latency}, defined as the end-to-end time delay between a developer action and the presentation of security feedback within the Integrated Development Environment (IDE). This includes the time required for context extraction, model inference, retrieval-augmented reasoning, and rendering of vulnerability findings or repair suggestions. While the system supports multiple interaction modes, the core task is the transformation of \emph{source code input} into \emph{actionable security feedback}. Accordingly, the latency requirement applies uniformly across inline detection, on-demand analysis, and repair suggestion workflows.

Responsiveness directly determines whether security assistance can be integrated naturally into the development process. Human–computer interaction research consistently shows that feedback delivered within a few seconds preserves a sense of flow and supports effective turn-taking, whereas longer delays disrupt concentration and reduce tool adoption \cite{card1991model, nielsen1994usability}. In the context of IDE-based development, developers expect near-immediate feedback comparable to other static diagnostics such as type errors or linting warnings. Excessive latency risks relegating security analysis to a background task that is ignored or deferred.

% As illustrated in Figure~\ref{fig:ide-interaction-example}, vulnerability
Vulnerability annotations and repair suggestions should appear promptly after a triggering event, such as saving a file or explicitly invoking an analysis command. Timely feedback enables developers to assess security implications while the relevant code context is still active, reducing cognitive load and improving remediation efficiency.

Unlike other requirements—such as detection accuracy (R1), contextual reasoning (R2), explainability (R3), repair quality (R4), or privacy preservation (R5)—usability in this thesis is scoped strictly to latency. This focus reflects the practical reality that even accurate and well-explained security findings are unlikely to be acted upon if they arrive too late to fit within normal development workflows. This concern is particularly relevant for local LLM-based systems, where inference time can be substantial compared to traditional static analysis.

Latency, however, is not an absolute property of the system alone. It is strongly influenced by the \textbf{hardware and deployment environment} on which the system operates. Dedicated accelerators such as GPUs can significantly reduce inference time, whereas CPU-only or resource-constrained environments typically incur higher latency. Additionally, factors such as model size, retrieval depth, and concurrency affect responsiveness. For this reason, all latency measurements must be reported together with the corresponding hardware profile, including processor type, available memory, and accelerator configuration. This ensures that usability claims are interpreted relative to realistic deployment scenarios rather than as hardware-agnostic performance guarantees.

For evaluation, usability is measured using the 95th percentile end-to-end (p95 E2E) latency across representative interaction scenarios. The p95 metric captures worst-case responsiveness experienced by users while remaining robust to isolated outliers. Separate latency measurements are reported for interactive inline detection and for more comprehensive, explicitly triggered analyses.

\begin{table}[h!]
\centering
\renewcommand{\arraystretch}{1.6}
\setlength{\tabcolsep}{12pt}
\begin{tabularx}{\textwidth}{|>{\centering\arraybackslash}m{3cm}|>{\arraybackslash}X|}
\hline
\textbf{Visual Score} & \textbf{Interpretation} \\
\hline
\centering\raisebox{0pt}{\tikz[baseline]{\filldraw[fill=black] (0,0) circle (0.4cm);}} 
& \textbf{High usability.} p95 end-to-end latency $\leq$ 2\,s for inline detection on the declared hardware profile. Interaction remains fluid and non-disruptive. \\
\hline
\centering\raisebox{0pt}{\tikz[baseline]{\filldraw[fill=black] (0,0) -- (90:0.4cm) arc (90:-90:0.4cm) -- cycle; \draw (0,0) circle (0.4cm);}}
& \textbf{Medium usability.} p95 end-to-end latency in $(2, 5)$\,s. Delay is noticeable but acceptable for security feedback that is not continuously triggered. \\
\hline
\centering\raisebox{0pt}{\tikz[baseline]{\draw (0,0) circle (0.4cm);}} 
& \textbf{No usability.} p95 end-to-end latency $>$ 5\,s for inline scenarios or frequent timeouts. Feedback is too slow for practical integration into the development workflow. \\
\hline
\end{tabularx}
\caption{Evaluation scale for R6: Usability (Latency).}
\label{tab:r6-usability}
\end{table}

In summary, usability in this framework is defined as responsiveness measured through end-to-end latency for IDE-integrated vulnerability detection and repair. Because latency is inherently dependent on hardware and deployment conditions, all usability evaluations must be contextualized by reporting the corresponding execution environment. This ensures that results are comparable, interpretable, and grounded in realistic usage scenarios.



\section{Related Work}
\label{sec:related-work}

This section reviews prior research relevant to LLM-based vulnerability detection and repair, with a focus on static analysis approaches, large language models for software security, retrieval-augmented generation, and privacy-preserving deployment. The discussion highlights both the strengths and limitations of existing work and positions the present thesis within the current research landscape.

\subsection{Traditional Static Application Security Testing}\vspace{-2mm}
Static Application Security Testing (SAST) tools have long been the primary means of detecting vulnerabilities during development. Rule-based analyzers and taint-analysis systems, such as Semgrep and CodeQL, identify predefined vulnerability patterns by statically inspecting source code \cite{semgrepDocs,codeqlDocs}. These tools provide deterministic and reproducible results, which makes them suitable for automated pipelines and compliance-driven environments.

From a research perspective, modern static analyzers build on foundational ideas such as abstract interpretation \cite{cousot1977abstract} and scalable bug-finding techniques \cite{engler2001bugs}. In practice, large-scale deployments demonstrate that static analysis can find real defects in industrial codebases at massive scale \cite{bessey2010billion}. Security-oriented static analysis has also been studied explicitly, including work that frames static analysis as an effective security engineering control when combined with secure development processes \cite{chess2004staticanalysis,howard2006sdl,mcgraw2006softwaresecurity}.

Despite these strengths, SAST tools can struggle with contextual reasoning and generalization. They frequently produce false positives when security-relevant patterns appear in benign contexts (e.g., input validated elsewhere, framework-enforced invariants) and may miss vulnerabilities that depend on semantic relationships or framework-specific behavior. Moreover, many SAST tools offer limited remediation guidance, requiring developers to interpret findings and manually design fixes. Empirical studies show that warning overload and poor actionability reduce adoption and remediation rates \cite{johnson2013don,christakis2016developers}.

These limitations motivate research into approaches that complement deterministic static analysis with more flexible reasoning and explanation generation. In this thesis, Code Guardian aims to preserve the determinism and localization benefits of IDE diagnostics while using LLM-based reasoning (grounded by security knowledge) to improve explanation quality and provide developer-controlled repair suggestions.


\subsection{LLM-Based Vulnerability Detection and Repair}\vspace{-2mm}
Recent advances in Large Language Models (LLMs) have prompted extensive investigation into their use for vulnerability detection and secure code generation. Empirical studies demonstrate that LLMs can identify security flaws and generate syntactically correct code across a range of programming tasks \cite{fu2023llmsec}. However, early evaluations reveal that LLM-generated code is often \emph{functionally correct yet insecure}, containing subtle vulnerabilities that are not immediately apparent \cite{pearce2022copilot,peng2025cweval}.

Several approaches attempt to address these shortcomings through fine-tuning or constrained decoding. SafeCoder introduces security-aware fine-tuning to bias models toward safer outputs \cite{he2023safecoder}, while constrained decoding techniques enforce security constraints during generation \cite{li2024constrained}. Although effective in controlled settings, these methods require curated training data, retraining effort, or auxiliary security models, limiting their adaptability and suitability for local, resource-constrained deployment.

Other systems combine LLMs with static analyzers to mitigate hallucinations and improve reliability. LLM Security Guard integrates static analysis feedback into LLM reasoning to strengthen vulnerability detection \cite{kavian2024llmsecguard}. Similarly, IRIS employs LLMs to assist static analysis by interpreting and refining vulnerability reports \cite{li2025iris}. While these hybrid approaches improve detection quality, many rely on centralized or cloud-based execution and do not explicitly address privacy or IDE-level integration.


\subsection{Retrieval-Augmented Generation for Secure Coding}\vspace{-2mm}
Retrieval-Augmented Generation (RAG) has emerged as a promising paradigm for grounding LLM outputs in external knowledge without retraining. Surveys on augmented language models highlight RAG's ability to improve factual accuracy and reduce hallucination by conditioning generation on retrieved context \cite{mialon2023augmented}.

In the domain of secure coding, several works leverage RAG to incorporate vulnerability knowledge, secure coding guidelines, and historical fixes. RESCUE proposes a hierarchical retrieval framework that combines vulnerability taxonomies with code examples to improve secure code generation \cite{shi2025rescue}. Evaluation results show improved security metrics compared to LLM-only baselines. However, RESCUE focuses on code generation rather than interactive vulnerability detection and assumes access to cloud-based or centralized resources.

Systematic literature reviews emphasize that while RAG improves detection consistency and security grounding, most existing approaches do not consider privacy-preserving local deployment or IDE-native interaction \cite{basic2025slr}. As a result, the practical applicability of RAG-based secure coding systems in industrial environments remains limited.


\subsection{Privacy-Preserving and IDE-Integrated Approaches}\vspace{-2mm}
Privacy concerns pose a significant barrier to adopting LLM-based security tools in real-world settings. Cloud-hosted assistants require transmitting proprietary source code to external servers, which is unacceptable in many regulated or security-sensitive environments. Recent surveys on LLMs in cybersecurity emphasize the growing demand for on-premise and locally executed solutions \cite{kaur2025cyberreview,gholami2024llmcyber}.

IDE-integrated security tools have been shown to improve developer engagement and remediation rates by providing feedback during active development rather than post hoc analysis. However, most IDE-based LLM assistants prioritize productivity features such as code completion and refactoring, with limited focus on security or privacy guarantees. Existing research rarely evaluates latency, resource usage, and reproducibility in local deployment scenarios, despite these factors being critical for practical adoption.


\subsection{Positioning of This Thesis}\vspace{-2mm}
In contrast to prior work, this thesis focuses explicitly on privacy-preserving vulnerability detection and repair using locally deployed LLMs integrated into Visual Studio Code. The proposed system combines retrieval-augmented reasoning with IDE-level context extraction to support both real-time and on-demand security analysis for JavaScript and TypeScript codebases. Unlike fine-tuned or cloud-dependent approaches, the system operates entirely offline, decouples security knowledge from model parameters, and is evaluated against established SAST baselines using reproducible benchmarks and quantitative metrics.

By addressing detection consistency, contextual reasoning, explainability, actionable repair, privacy preservation, and usability within a unified framework, this work aims to bridge the gap between academic advances in LLM-based security analysis and the practical requirements of real-world software development workflows.

