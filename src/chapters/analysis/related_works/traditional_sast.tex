Static Application Security Testing (SAST) tools have long been the primary means of detecting vulnerabilities during development. Rule-based analyzers and taint-analysis systems, such as Semgrep and CodeQL, identify predefined vulnerability patterns by statically inspecting source code \cite{semgrepDocs,codeqlDocs}. These tools provide deterministic and reproducible results, which makes them suitable for automated pipelines and compliance-driven environments.

From a research perspective, modern static analyzers build on foundational ideas such as abstract interpretation \cite{cousot1977abstract} and scalable bug-finding techniques \cite{engler2001bugs}. In practice, large-scale deployments demonstrate that static analysis can find real defects in industrial codebases at massive scale \cite{bessey2010billion}. Security-oriented static analysis has also been studied explicitly, including work that frames static analysis as an effective security engineering control when combined with secure development processes \cite{chess2004staticanalysis,howard2006sdl,mcgraw2006softwaresecurity}.

Despite these strengths, SAST tools can struggle with contextual reasoning and generalization. They frequently produce false positives when security-relevant patterns appear in benign contexts (e.g., input validated elsewhere, framework-enforced invariants) and may miss vulnerabilities that depend on semantic relationships or framework-specific behavior. Moreover, many SAST tools offer limited remediation guidance, requiring developers to interpret findings and manually design fixes. Empirical studies show that warning overload and poor actionability reduce adoption and remediation rates \cite{johnson2013don,christakis2016developers}.

These limitations motivate research into approaches that complement deterministic static analysis with more flexible reasoning and explanation generation. In this thesis, Code Guardian aims to preserve the determinism and localization benefits of IDE diagnostics while using LLM-based reasoning (grounded by security knowledge) to improve explanation quality and provide developer-controlled repair suggestions.
