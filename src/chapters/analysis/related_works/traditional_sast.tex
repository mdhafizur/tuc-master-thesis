Static Application Security Testing (SAST) tools have long been the primary means of detecting vulnerabilities during development. Rule-based analyzers and taint-analysis systems, such as Semgrep and CodeQL, identify predefined vulnerability patterns by statically inspecting source code. These tools provide deterministic and reproducible results, which makes them suitable for automated pipelines and compliance-driven environments.

However, multiple studies report that traditional SAST tools struggle with contextual reasoning and generalization. They frequently produce false positives when security-relevant patterns appear in benign contexts and may miss vulnerabilities that depend on semantic relationships or framework-specific behavior \cite{sheng2025survey,li2025iris}. Moreover, most SAST tools offer limited remediation guidance, requiring developers to interpret findings and manually design fixes. These limitations motivate research into approaches that combine static analysis with more flexible reasoning mechanisms.
