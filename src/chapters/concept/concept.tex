This chapter presents the conceptual design of a privacy-preserving, retrieval-augmented vulnerability detection and repair system integrated into Visual Studio Code. The core objective is to assist developers in identifying and remediating security vulnerabilities in JavaScript and TypeScript code using locally deployed Large Language Models (LLMs), without transmitting source code to external services.

The design is motivated by three critical shortcomings observed in existing work. First, traditional Static Application Security Testing (SAST) tools rely on predefined rules and struggle with contextual reasoning, producing false positives when security-relevant patterns appear in benign contexts \cite{sheng2025survey,li2025iris}. Second, cloud-based LLM assistants require transmitting proprietary source code to external servers, which is unacceptable in regulated or security-sensitive environments \cite{kaur2025cyberreview,gholami2024llmcyber}. Third, current approaches provide limited explainability and repair guidance, making it difficult for developers to understand and act upon detected vulnerabilities \cite{johnson2013don,christakis2016developers}.

The proposed system addresses these limitations through a modular architecture that enforces separation of concerns while preserving end-to-end coherence. The design directly supports the six requirements established in Section~\ref{sec:requirements}: detection consistency is enforced through retrieval-augmented grounding (R1), contextual reasoning is enabled through code analysis and data flow understanding (R2), intermediate artifacts and explanations provide transparency (R3), concrete repair suggestions are generated based on security knowledge (R4), all processing occurs locally without external data transmission (R5), and latency is optimized for IDE-integrated workflows (R6).

The remainder of this chapter is organized as follows. Section~\ref{sec:concept-derivation} derives the conceptual design from the analysis results, explaining how each requirement motivates specific architectural decisions. Section~\ref{sec:system-components} describes the core components that comprise the system, detailing their responsibilities, inputs, outputs, and interactions. Section~\ref{sec:detection-strategies} presents detection and analysis workflows that illustrate how the system operates in different scenarios. Section~\ref{sec:system-architecture} provides the high-level system architecture and design, including context diagrams, component views, and process flows. Detailed implementation and experimental validation are deferred to Chapters~\ref{chap:implementation} and~\ref{chap:evaluation}, respectively.

\section{Concept Derivation from Analysis Results}
\label{sec:concept-derivation}

The conceptual design of the proposed vulnerability detection system is derived systematically from the six requirements identified in Section~\ref{sec:requirements} and the gaps observed in existing approaches reviewed in Section~\ref{sec:related-work}. This section traces how each architectural decision directly addresses specific limitations in current vulnerability detection systems, establishing the rationale for a local, RAG-augmented, IDE-integrated design.

\subsection{From Cloud-Based to Privacy-Preserving Local Deployment}
\label{subsec:cloud-to-local}

The analysis in Section~\ref{sec:related-work} revealed that most LLM-based vulnerability detection systems rely on cloud-hosted models or external APIs, requiring transmission of source code to remote servers. This poses unacceptable privacy risks in industrial, governmental, and regulated settings where source code contains proprietary logic, intellectual property, or sensitive business information \cite{kaur2025cyberreview,gholami2024llmcyber}.

These observations directly motivate the decision to deploy all components locally within the developer's environment. By running LLMs, retrieval systems, and analysis components entirely on local hardware, the system ensures that no source code, intermediate representations, or analysis results are transmitted beyond the developer's machine. This design directly satisfies R5 (privacy-preserving operation) and enables offline operation, making the system suitable for air-gapped or security-sensitive development environments.

The trade-off for local deployment is increased latency compared to cloud-based systems with dedicated accelerators. However, by carefully optimizing model selection, retrieval strategies, and context management, the system can achieve acceptable response times on standard developer hardware, addressing R6 (usability and responsiveness).

\subsection{Retrieval-Augmented Generation for Security Knowledge Grounding}
\label{subsec:rag-security}

Requirement R1 demands consistent and accurate vulnerability detection across repeated analyses. Pure generative LLM approaches suffer from hallucination and inconsistent classifications, as shown in prior studies \cite{sheng2025survey,li2025everything}. Similarly, R2 requires context-aware reasoning that goes beyond surface-level pattern matching to understand data flow, control flow, and API semantics.

The system addresses these requirements through Retrieval-Augmented Generation (RAG), which grounds vulnerability detection in a locally maintained knowledge base of security information. This knowledge base includes:
\begin{itemize}
\item \textbf{Vulnerability descriptions} from the Common Weakness Enumeration (CWE) taxonomy
\item \textbf{Secure coding guidelines} for JavaScript and TypeScript
\item \textbf{Historical vulnerability examples} with explanations and fixes
\item \textbf{Framework-specific security patterns} for common libraries
\end{itemize}

During analysis, relevant security knowledge is retrieved based on semantic similarity to the code under examination and provided as context to the LLM. This design reduces reliance on purely generative reasoning, improves detection consistency by anchoring outputs to established security knowledge, and enables the knowledge base to evolve independently of model parameters. By decoupling security knowledge from the model, the system supports continuous updates to vulnerability information without requiring model retraining.

\subsection{IDE Integration for In-Context Security Assistance}
\label{subsec:ide-integration}

Traditional SAST tools operate as separate processes in CI/CD pipelines, providing feedback only after code is committed. This delayed feedback loop increases cognitive load and reduces remediation rates, as developers must context-switch between writing code and reviewing security findings \cite{johnson2013don,christakis2016developers}.

The proposed system integrates directly into Visual Studio Code as an extension, providing security analysis during active development. This design supports two interaction modes:
\begin{itemize}
\item \textbf{Inline detection}: Real-time vulnerability annotations triggered by file save events, providing immediate feedback similar to type errors or linting warnings
\item \textbf{On-demand analysis}: Explicit analysis commands for comprehensive security audits of selected code regions or entire files
\end{itemize}

IDE integration directly addresses R3 (explainability and transparency) by enabling rich, interactive presentation of vulnerability findings, explanations, and repair suggestions within the developer's familiar environment. It also supports R4 (actionable repair suggestions) by allowing developers to preview, review, and apply suggested fixes with minimal friction.

\subsection{Modular Component Architecture}
\label{subsec:modular-architecture}

The analysis in Section~\ref{sec:related-work} showed that monolithic vulnerability detection systems lack transparency: when detection fails or produces unexpected results, it is difficult to diagnose whether the error originated in context extraction, retrieval, reasoning, or explanation generation.

The proposed system decomposes vulnerability detection into four core components, each with clearly defined responsibilities:
\begin{enumerate}
\item \textbf{Context Extraction}: Parses source code and extracts relevant context (function definitions, imports, data flow)
\item \textbf{Knowledge Retrieval}: Queries the local security knowledge base to retrieve relevant vulnerability information
\item \textbf{Vulnerability Detection}: Analyzes code context using the LLM, grounded in retrieved security knowledge
\item \textbf{Repair Generation}: Produces concrete, context-appropriate fixes for detected vulnerabilities
\end{enumerate}

This modular design enables component-level analysis, isolated improvement, and transparent error diagnosis. Each component can be tested, optimized, and replaced independently without destabilizing the overall architecture. Intermediate outputs (extracted context, retrieved knowledge, detection reasoning) remain visible for debugging and validation, supporting transparency and reproducibility.

\subsection{Context-Aware Analysis with Code Understanding}
\label{subsec:context-aware-analysis}

Requirement R2 demands that the system correctly identify vulnerabilities based on semantic and structural context rather than syntactic patterns alone. Surface-level pattern matching produces false positives when secure code resembles vulnerable patterns, and false negatives when vulnerabilities depend on data flow or control flow relationships.

The system addresses this through multi-level context extraction that captures:
\begin{itemize}
\item \textbf{Syntactic context}: Abstract Syntax Tree (AST) representations of code structure
\item \textbf{Semantic context}: Variable definitions, function signatures, import statements, and type information
\item \textbf{Data flow context}: Traces of how untrusted inputs flow through the code to sensitive operations
\item \textbf{Control flow context}: Conditional branches, validation checks, and mitigation logic
\end{itemize}

This context is provided to the LLM alongside retrieved security knowledge, enabling reasoning about whether observed code patterns constitute genuine vulnerabilities or are mitigated by contextual safeguards. The design directly supports R2 by enabling the system to distinguish between vulnerable and benign code that appears syntactically similar.

\subsection{Explainability Through Structured Reasoning}
\label{subsec:explainability-design}

Requirement R3 demands transparent explanations that link detected vulnerabilities to concrete code regions and recognized security principles. Opaque "black box" detection undermines developer trust and makes it difficult to validate findings or apply fixes correctly \cite{johnson2013don,christakis2016developers}.

The system enforces explainability through structured output generation, where vulnerability reports must include:
\begin{itemize}
\item \textbf{Vulnerability classification}: CWE category and severity level
\item \textbf{Code localization}: Specific lines or code fragments contributing to the vulnerability
\item \textbf{Reasoning explanation}: Description of why the code is vulnerable (e.g., untrusted input, missing validation, dangerous API usage)
\item \textbf{Security principle violated}: Reference to secure coding guidelines or CWE descriptions
\end{itemize}

By grounding explanations in retrieved security knowledge and enforcing structured output formats, the system produces consistent, verifiable, and actionable vulnerability reports. This design directly supports R3 and improves developer trust by making detection reasoning transparent and auditable.

This section has shown how each architectural decision maps to the requirements established in Chapter~\ref{chap:analysis}. The local, RAG-augmented, IDE-integrated design is not an arbitrary choice, but a direct consequence of the limitations observed in existing work and the operational constraints of real-world, privacy-sensitive development environments. The next section details the core components that instantiate this design, specifying their inputs, outputs, and processing logic.


\section{System Components}
\label{sec:system-components}

The Code Guardian prototype is organized into a small set of components that map directly to the requirements from Chapter~\ref{chap:analysis}. The system is implemented as a VS Code extension that (i) extracts an appropriate analysis scope from the editor, (ii) invokes a local LLM for vulnerability analysis, (iii) optionally augments prompts with locally retrieved security knowledge (RAG), and (iv) renders findings and fix suggestions using IDE-native UI elements.

\subsection{Context Extraction Component}
\label{subsec:context-extraction}

The context extraction component determines \emph{which} code is analyzed for a given interaction mode and provides the analyzer with enough metadata to localize findings in the editor.

\textbf{Scopes.} Code Guardian supports multiple scopes with different latency and completeness characteristics:
\begin{itemize}
    \item \textbf{Function scope (real-time)} extracts the innermost function-like block at the cursor position (function declarations, arrow functions, methods, constructors). This is the default for continuous feedback because it keeps the prompt small.
    \item \textbf{File scope (on-demand)} analyzes the full current document and returns a comprehensive set of findings as diagnostics.
    \item \textbf{Selection scope (interactive)} sends a selected region (or current line) into a webview-based analysis view that supports follow-up questions.
    \item \textbf{Workspace scope (batch)} scans all JS/TS files and aggregates results into a security dashboard.
\end{itemize}

\textbf{AST-based function extraction.} Function scope extraction is implemented by parsing the current document with the TypeScript compiler API and selecting the smallest enclosing \texttt{FunctionLikeDeclaration} around the cursor. The extractor returns both the extracted snippet and its start-line offset (0-based) in the original document. This offset enables precise mapping of model-reported line numbers back to VS Code diagnostic ranges.

\textbf{Design rationale.} Function scoping is a practical mechanism to keep latency predictable for real-time use (R6), while the line-offset mapping improves explainability by ensuring that findings point to the correct source locations (R3). Deeper program context (imports, cross-file flows) is not extracted explicitly in the current prototype and is treated as a key future-work direction.

\subsection{Knowledge Retrieval Component (RAG)}
\label{subsec:knowledge-retrieval}

The retrieval component implements Retrieval-Augmented Generation (RAG) to ground LLM reasoning in local security knowledge \cite{lewis2020rag,karpukhin2020dpr}. In Code Guardian, retrieval is implemented by a dedicated \texttt{RAGManager} that maintains a local knowledge base and a persistent vector index.

\textbf{Knowledge base contents.} The knowledge base is populated from a mix of curated and fetched \emph{public} security metadata:
\begin{itemize}
    \item \textbf{CWE-aligned entries} describing common weakness patterns and mitigations \cite{mitreCWE}.
    \item \textbf{OWASP Top 10 guidance} for recurring web vulnerability categories \cite{owaspTop10_2021}.
    \item \textbf{CVE/NVD summaries} retrieved from the NVD API (descriptions and references) \cite{nistNVD,mitreCVE}.
    \item \textbf{JavaScript ecosystem advisories} for dependency and platform-specific risks.
\end{itemize}
Knowledge artifacts are cached on disk and reused across sessions. When network access is unavailable, the system falls back to a small baseline knowledge bundle so retrieval remains functional (with reduced coverage).

\textbf{Indexing and retrieval.} Knowledge entries are chunked using a recursive splitter (chunk size 1000, overlap 200) and embedded locally through an Ollama-served embedding model. Embeddings are stored in a local HNSW vector index \cite{malkov2018hnsw} via LangChain tooling \cite{langchainDocs}. At query time, the top-$k$ most similar chunks are retrieved (default $k=3$) and injected into the prompt as an explicit ``relevant security knowledge'' section.

\textbf{Privacy boundary.} Retrieval and embeddings are executed locally. Optional knowledge refresh operations fetch only public vulnerability metadata; user source code is not transmitted to those endpoints (R5).

\subsection{Vulnerability Detection Component}
\label{subsec:vulnerability-detection}

The vulnerability detection component performs local LLM inference and returns findings in a format suitable for IDE integration.

\textbf{Structured-output analyzer for diagnostics.} For inline diagnostics, Code Guardian enforces a strict JSON-only output contract to reduce ambiguity and parsing failures. Each finding includes:
\begin{itemize}
    \item \textbf{message}: a short description of the issue.
    \item \textbf{startLine}/\textbf{endLine}: 1-based line indices relative to the analyzed snippet.
    \item \textbf{suggestedFix} (optional): a replacement string that can be offered as a quick fix.
\end{itemize}
If no issues are found, the model must return an empty array \texttt{[]}. The analyzer performs defensive parsing by stripping code fences and extracting the first JSON array substring if the model emits additional text.

\textbf{Failure handling and caching.} Transient inference failures (timeouts, local server warm-up) are handled through retry with exponential backoff. To reduce repeated inference on unchanged code, results are cached with an LRU-style strategy keyed by a hash of (code, model), improving responsiveness during iterative edits.

\textbf{Interactive analysis mode.} In addition to the structured diagnostics flow, Code Guardian includes a webview-based analysis view for selected code. This mode uses Markdown-formatted responses and supports follow-up Q\&A. When enabled, the RAG manager can be used to enrich the system prompt and user prompt for this interactive workflow.

\subsection{Repair Generation Component}
\label{subsec:repair-generation}

In the current prototype, repair generation is implemented as an \emph{optional field} in the vulnerability report rather than as a separate autonomous patching system. When the analyzer includes a \texttt{suggestedFix}, the IDE integration layer exposes it as a quick fix action. Applying a fix is always user-initiated and can be reverted via the editor undo stack (R4). The system does not automatically validate functional correctness of repairs; this is treated as a major future improvement area.

\subsection{IDE Integration Layer}
\label{subsec:ide-integration}

The IDE integration layer connects the analysis pipeline to VS Code’s APIs \cite{vscodeExtensionApi} and determines when analyses run and how results are presented.

\textbf{Triggers.} The extension supports both automatic and manual triggers:
\begin{itemize}
    \item \textbf{Debounced real-time analysis} on document changes (default debounce: 800\,ms) for JavaScript/TypeScript documents.
    \item \textbf{Manual file analysis} via a command palette command, with a size guard to avoid excessively large prompts.
    \item \textbf{Interactive selection analysis} and \textbf{contextual Q\&A} views implemented as WebViews.
    \item \textbf{Workspace scanning} that aggregates results into a dashboard view.
\end{itemize}

\textbf{Presentation.} Findings are rendered as VS Code diagnostics (squiggles, Problems panel, hover tooltips). Suggested repairs are attached to diagnostics and exposed as a quick fix action.

\subsection{Supporting Subsystems}
\label{subsec:supporting-subsystems}

Several additional subsystems are important for usability and reproducibility:
\begin{itemize}
    \item \textbf{Model management} queries available local Ollama models, filters for suitable code models, and supports runtime model switching.
    \item \textbf{Analysis cache} is a bounded LRU cache (100 entries, 30-minute TTL) with user-visible hit/miss statistics.
    \item \textbf{Workspace dashboard} computes a coarse security score using issue density and keyword-based severity heuristics and visualizes the distribution across files.
    \item \textbf{Vulnerability data manager} caches public metadata (e.g., OWASP/CWE/CVE-derived entries) with a time-based expiry to support offline reuse after updates.
\end{itemize}

This section has described the core components that constitute Code Guardian. The next section describes how these components are orchestrated into workflows for real-time feedback, on-demand inspection, and batch scans.

\section{Detection Workflows}
\label{sec:detection-strategies}

While the component architecture defines what responsibilities exist in the system, IDE usability depends on how these components are orchestrated in concrete workflows. Code Guardian supports several workflows that trade off latency, context breadth, and output structure. This section describes the main workflows implemented in the prototype and relates them to requirements R1--R6.

\subsection{Real-Time Function-Level Diagnostics}
\label{subsec:inline-detection}

The real-time workflow provides continuous feedback while a developer edits JavaScript/TypeScript code. The key goal is to deliver timely, IDE-native warnings without disrupting flow (R6).

\textbf{Trigger.} The extension listens to document change events for JavaScript and TypeScript documents. Analysis is \emph{debounced} (default: 800\,ms) so the model is not invoked on every keystroke.

\textbf{Scope and guards.} When the debounce fires, Code Guardian extracts the innermost enclosing function at the cursor position and analyzes only that snippet. A size guard skips unusually large functions (default: 2000 characters) to bound worst-case latency and avoid overloading the local model.

\textbf{Structured output for diagnostics.} The analyzer is prompted to return a strict JSON array of issues. Each issue contains a message, a line range, and an optional fix string. The diagnostic adapter maps the snippet-relative, 1-based line numbers to VS Code ranges using the extractor’s line offset and clamps ranges to valid document bounds. This enables stable rendering in the Problems panel, editor squiggles, and hover tooltips (R3).

\textbf{Trade-offs.} Function-level analysis improves responsiveness, but it can miss vulnerabilities whose evidence lies outside the current scope (e.g., validation in a different module). This limitation motivates the on-demand and workspace workflows and is addressed further in the future-work chapter.

\subsection{On-Demand File Diagnostics}
\label{subsec:comprehensive-analysis}

The file workflow provides broader coverage when a developer explicitly requests a deeper scan.

\textbf{Trigger.} A command palette action runs analysis over the full active document.

\textbf{Scope guard.} To avoid generating excessively large prompts, the prototype skips very large files (default: 20{,}000 characters) and warns the user instead.

\textbf{Output.} Findings are surfaced through the same structured diagnostics pipeline as the real-time workflow. This keeps the UI consistent, and it ensures that file scans can be reviewed in the Problems panel and navigated using standard IDE tooling.

\subsection{Interactive Analysis and Follow-up Q\&A}
\label{subsec:interactive-analysis}

Some security questions require richer explanations than a single diagnostic message. Code Guardian therefore includes webview-based interaction modes.

\textbf{Selection analysis view.} The user can analyze a selected region (or the current line) and inspect the response in a dedicated analysis panel. This mode supports follow-up questions and streams Markdown-formatted responses for readability. Because it is user-initiated and not continuously triggered, it can tolerate higher latency than real-time diagnostics.

\textbf{Contextual Q\&A view.} The contextual Q\&A panel allows the user to select files and folders as context and ask security questions about that subset of the workspace. The extension reads the selected files locally and sends their contents only to the local LLM backend, preserving the no-exfiltration objective for source code.

\textbf{RAG usage.} When enabled, the RAG manager can enrich prompts for these interactive workflows by injecting retrieved security knowledge snippets (CWE/OWASP/CVE-derived guidance). Retrieval and embeddings are performed locally; optional knowledge refresh operations fetch only public metadata.

\subsection{Workspace Scan and Security Dashboard}
\label{subsec:workspace-scan}

The workspace workflow supports periodic audits and prioritization across a repository.

\textbf{File discovery and bounds.} The scanner enumerates \texttt{*.js}, \texttt{*.jsx}, \texttt{*.ts}, and \texttt{*.tsx} files in the workspace while excluding dependency folders such as \texttt{node\_modules}. Very large files are skipped (default: $>$500\,KB) to keep runtime bounded.

\textbf{Aggregation and scoring.} Scan results are aggregated into a dashboard WebView. In addition to listing per-file issues, the dashboard computes a coarse security score based on issue density (issues per KLOC) and a keyword-based severity heuristic. This score is intended for prioritization rather than as a formal risk metric.

\textbf{Developer interaction.} The dashboard supports opening affected files directly from the report and rerunning scans. This workflow complements real-time diagnostics by helping developers understand which parts of the codebase concentrate the most findings.

\subsection{Repair Suggestion Workflow}
\label{subsec:repair-workflow}

When the analyzer returns a \texttt{suggestedFix} for an issue, Code Guardian exposes it as a VS Code quick fix. Applying repairs is always user-initiated and integrates with the undo stack. This preserves developer control (R4) and reduces the risk of unintended behavioral changes from automatically applied patches.

\subsection{Workflow Selection in Practice}
\label{subsec:workflow-integration}

In typical use, workflows complement each other:
\begin{enumerate}
    \item Developers receive continuous feedback via debounced, function-level diagnostics while writing code.
    \item Before committing or during review, developers run file scans for broader coverage.
    \item For ambiguous findings or design-level questions, developers use the interactive analysis and contextual Q\&A views to obtain richer explanations and mitigation guidance.
    \item Periodically, workspace scans provide an aggregate view that supports prioritization and remediation planning.
\end{enumerate}

Together, these workflows operationalize the core idea of Code Guardian: privacy-preserving local analysis combined with IDE-native feedback and developer-controlled remediation.

\section{System Architecture and Design}
\label{sec:system-architecture}

This section presents the high-level architecture of Code Guardian using C4-style views (context, containers, and process). The goal is to make the privacy boundary and the main data flows explicit: code stays on-device, local inference is performed through a localhost LLM backend, and retrieval (when enabled) is backed by a local knowledge base and vector index.

\subsection{Context View: System Boundary and External Interactions}
\label{subsec:context-view}

Figure~\ref{fig:c4-context} positions Code Guardian within its operational environment. The primary actor is the developer working inside Visual Studio Code. The system executes in the VS Code extension host and communicates with a local LLM backend (Ollama) running on the same machine \cite{ollamaDocs}.

\textbf{Local assets.} The core local assets are:
\begin{itemize}
    \item \textbf{Workspace source code} (JavaScript/TypeScript) being edited.
    \item \textbf{Local LLM runtime} (Ollama) used for analysis and (optionally) embeddings.
    \item \textbf{Local security knowledge base} and \textbf{vector index} used for retrieval when RAG is enabled \cite{langchainDocs,malkov2018hnsw}.
    \item \textbf{Local caches} for analysis results and vulnerability metadata.
\end{itemize}

\textbf{Optional external data.} Code Guardian may optionally fetch \emph{public vulnerability metadata} to refresh the knowledge base (e.g., CVE records from the NVD API). This traffic does not include user source code and can be disabled by running in offline mode. After a refresh, cached knowledge can be reused without network access. This separation preserves the core privacy goal (no source-code exfiltration) while still allowing the knowledge base to evolve over time.

\begin{figure}[H]
  \centering
  \includegraphics[width=1\linewidth]{images/c4_context.drawio.pdf}
  \caption{Context diagram: Code Guardian runs locally in the VS Code extension host and calls a local LLM backend. A local knowledge base supports retrieval; optional refresh operations may fetch public vulnerability metadata, but source code remains local.}
  \label{fig:c4-context}
\end{figure}

\subsection{Container View: Internal Component Structure}
\label{subsec:container-view}

Figure~\ref{fig:c4-container} summarizes the main internal containers and their responsibilities.

\textbf{VS Code extension host.} The extension is responsible for registering editor triggers, extracting analysis scopes, and rendering results using VS Code diagnostics and WebViews \cite{vscodeExtensionApi}. It provides commands for file analysis, selection analysis, contextual Q\&A, model selection, cache inspection, and workspace scanning.

\textbf{Structured diagnostics pipeline.} For real-time and file-level diagnostics, the extension invokes a JSON-only analyzer that returns a list of issues with line ranges and optional fixes. These results are mapped into VS Code diagnostics and quick fixes. A bounded analysis cache reduces redundant LLM calls for unchanged snippets.

\textbf{Interactive analysis pipeline.} For selection analysis and contextual Q\&A, the extension opens a WebView panel and streams Markdown-formatted responses from the local model. This mode supports conversational follow-up and can incorporate retrieved knowledge when RAG is enabled.

\textbf{RAG manager and data manager.} When enabled, the RAG manager maintains a local knowledge base (serialized entries) and a persistent vector store. A vulnerability data manager refreshes public metadata (CWE-/OWASP-aligned curated entries and optional CVE/advisory sources) and caches results on disk. The vector store is rebuilt or updated based on the knowledge base content.

\begin{sidewaysfigure}
  \centering
  \includegraphics[width=1\linewidth]{images/c4_container.drawio.pdf}
  \caption{Container diagram: The VS Code extension orchestrates context extraction, local LLM analysis, optional retrieval augmentation, and IDE-native rendering. Knowledge and caches are stored locally; optional knowledge refreshes fetch only public metadata.}
  \label{fig:c4-container}
\end{sidewaysfigure}

\subsection{Process View: End-to-End Workflows}
\label{subsec:process-view}

Figure~\ref{fig:bpmn} illustrates the dynamic execution flow for the main workflows.

\textbf{Real-time diagnostics (function scope).}
\begin{enumerate}
    \item A JavaScript/TypeScript document change event occurs in the active editor.
    \item After a debounce interval (800\,ms), the extension extracts the innermost enclosing function at the cursor.
    \item If the extracted scope is within size limits, the local analyzer is invoked and required to return a JSON array of findings.
    \item Findings are parsed defensively, cached, mapped to document ranges, and rendered as diagnostics. Optional \texttt{suggestedFix} strings are surfaced as quick fixes.
\end{enumerate}

\textbf{On-demand file diagnostics.}
\begin{enumerate}
    \item The developer invokes the ``analyze full file'' command.
    \item The full document text is analyzed (subject to a size guard).
    \item Findings are mapped to diagnostics and rendered in the editor.
\end{enumerate}

\textbf{Interactive analysis (selection) and contextual Q\&A.}
\begin{enumerate}
    \item The developer selects code (or context files/folders) and asks a security question.
    \item The extension collects the selected context locally and opens a WebView panel.
    \item The local model streams Markdown-formatted responses; follow-up questions extend the conversation history.
    \item When enabled, retrieved security knowledge can be injected into prompts to ground explanations.
\end{enumerate}

\textbf{Workspace scan and dashboard.}
\begin{enumerate}
    \item The developer starts a workspace scan from the command palette.
    \item The scanner enumerates JS/TS files, excluding dependencies, and skips very large files.
    \item Each file is analyzed locally; issues are aggregated and summarized by severity heuristics and issue density.
    \item Results are shown in a dashboard WebView, which can open files directly and trigger rescans.
\end{enumerate}

\textbf{Privacy considerations in the process view.} Across all workflows, source code is sent only to the local LLM backend on \texttt{localhost}. Optional knowledge refresh operations fetch only public metadata and are cached; disabling refresh yields a fully offline analysis mode.

\begin{sidewaysfigure}
  \centering
  \includegraphics[width=\linewidth]{images/bpmn_process_flow.pdf}
  \caption{Process view: debounced real-time function analysis, on-demand file analysis, interactive analysis/Q\&A, and workspace scanning. The privacy boundary is maintained by keeping code on-device and using local inference.}
  \label{fig:bpmn}
\end{sidewaysfigure}

These architecture views make explicit how Code Guardian operationalizes the conceptual design: modular responsibilities, local inference as the default deployment, optional retrieval grounding, and IDE-native presentation with developer-controlled remediation.
