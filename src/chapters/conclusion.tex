This chapter summarizes the contributions of this thesis, highlights key findings, discusses limitations, and outlines implications for practice. The goal of the thesis was to design and implement a privacy-preserving vulnerability detection and repair assistant for Visual Studio Code that leverages locally deployed LLMs and retrieval-augmented grounding without transmitting source code to external services.

\section{Summary of Contributions}

\textbf{Code Guardian: an IDE-integrated, local security assistant.}
This thesis delivers \textbf{Code Guardian}, a VS Code extension that performs on-device vulnerability analysis for JavaScript and TypeScript projects. Findings are presented using IDE-native diagnostics, and optional quick fixes provide repair suggestions while keeping developers in full control of code changes.

\textbf{Privacy-preserving LLM inference with optional RAG.}
All code analysis runs locally via Ollama. To improve grounding, the system optionally augments prompts with locally retrieved security knowledge (CWE/OWASP/CVE guidance) using a local vector index and local embeddings. This architecture supports explainability and consistency while preserving the no-exfiltration requirement.

\textbf{Practical performance mechanisms.}
To remain usable during development, Code Guardian combines debounced triggers, function-level scoping for real-time use, and caching of repeated analyses. These mechanisms reduce unnecessary inference calls and support responsive IDE feedback.

\textbf{Reproducible evaluation harness.}
The prototype includes a curated benchmark of security test cases and a local evaluation script for comparing models and configurations using standard detection metrics, parse robustness, and latency.

\section{Limitations}

\textbf{Scope limits and contextual depth.}
While the system can flag common vulnerability patterns, deep semantic reasoning across files (e.g., source-to-sink flows spanning modules) is limited by the analysis scope and the absence of full static data-flow analysis.

\textbf{Evaluation representativeness.}
The curated dataset is intentionally small and human-auditable, but it does not fully reflect the diversity and ambiguity of real-world codebases. Results should therefore be interpreted as indicative rather than definitive.

\textbf{Repair correctness.}
Repair suggestions are model-generated and may affect behavior beyond security hardening. The system mitigates this by requiring explicit user review, but comprehensive functional validation remains outside the scope of the extension.

\section{Conclusion}

Privacy-preserving secure coding assistance is feasible within the IDE when local LLM inference is combined with careful prompt structuring, optional retrieval grounding, and developer-controlled remediation workflows. Code Guardian demonstrates that locally deployed LLMs can provide actionable vulnerability detection and repair suggestions without transmitting source code off-device, while remaining compatible with interactive development constraints.
