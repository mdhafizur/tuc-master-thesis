Privacy concerns pose a significant barrier to adopting LLM-based security tools in real-world settings. Cloud-hosted assistants require transmitting proprietary source code to external servers, which is unacceptable in many regulated or security-sensitive environments \cite{gdpr2016}. Beyond policy constraints, research has demonstrated concrete privacy risks in large language models, including memorization and extraction of training data \cite{carlini2021extracting} and inference attacks that raise questions about model confidentiality and data exposure \cite{shokri2017membership}. For security tooling, these risks motivate keeping source code and analysis artifacts strictly local.

The IDE context introduces additional security considerations. Because the assistant processes attacker-controlled inputs (e.g., comments, strings, dependency code), prompt-injection and tool-manipulation attacks become relevant; OWASP explicitly catalogs such risks for LLM applications \cite{owaspLLMTop10_2023}. These concerns motivate designs that treat code as data, constrain outputs to machine-checkable schemas, and isolate retrieval sources to curated security knowledge.

IDE-integrated security tools can improve developer engagement and remediation rates by providing feedback during active development rather than post hoc analysis \cite{johnson2013don,christakis2016developers}. However, many IDE assistants prioritize productivity features such as code completion and refactoring, with limited focus on security guarantees, reproducibility, or privacy boundaries. In practice, local deployment and deterministic interaction contracts become critical: users must understand what data is processed, where it is processed, and how results may vary across models and runs.
