\section{Experimental Setup}
\label{sec:eval-setup}

All experiments are executed locally using the Code Guardian evaluation harness in \path{code-guardian-extension/evaluation/evaluate-models.js}. The harness iterates over the curated dataset described in Section~\ref{sec:eval-dataset} and invokes Ollama with a strict JSON-only system prompt. Model decoding is configured for low randomness (\texttt{temperature=0.1}) to reduce output variance and improve parsing stability.

\subsection*{Response Schema for Scoring}

For evaluation purposes, the harness requests a richer structured output than the minimal in-editor diagnostics flow. In particular, each finding includes a vulnerability \texttt{type} string and a coarse \texttt{severity} level in addition to location and an optional fix. This enables type-level scoring (precision/recall) and per-category analysis. In the extension UI, vulnerability categories may appear within the message text and are used for downstream aggregation (e.g., in the workspace dashboard); a future refinement is to surface \texttt{type} and \texttt{severity} as first-class fields in diagnostics.

\subsection*{Compared Configurations}

Two configurations are evaluated quantitatively:
\begin{itemize}
  \item \textbf{LLM-only:} JSON-only analyzer without retrieval context.
  \item \textbf{LLM+RAG:} retrieval-augmented prompting with static security snippets (\texttt{k=5}).
\end{itemize}

Where relevant, results can also be stratified by model size or model family to study the trade-off between accuracy and latency.

\subsection*{Models and Hardware}

The reported run evaluated five local Ollama models: \texttt{gemma3:1b}, \texttt{gemma3:4b}, \texttt{qwen3:4b}, \texttt{qwen3:8b}, and \texttt{CodeLlama:latest}. The execution environment was macOS (Darwin 25.3.0, arm64) on Apple M4 Max (16 CPU cores, 64\,GB RAM), Node.js v20.19.5, and Ollama 0.17.1.

\subsection*{Runtime Parameters}

The evaluation harness enforces a per-request timeout (\texttt{timeoutMs=30000}) and limits generation length (\texttt{num\_predict=1000}). A request delay of \texttt{500\,ms} is inserted between calls to reduce burstiness on developer hardware. Runs were executed sequentially with no warm-up and no retries.

\subsection*{Baseline Tool Configuration}

To contextualize LLM and LLM+RAG behavior, three SAST baselines are executed through the same harness: Semgrep, CodeQL, and ESLint with \texttt{eslint-plugin-security}. Each baseline tool runs on a temporary workspace created by the harness that contains one \texttt{.js} or \texttt{.ts} file per snippet. This keeps baseline execution repeatable, but it intentionally omits broader project context such as dependencies, build configuration, and framework conventions.

Semgrep is run with the Semgrep Registry packs \texttt{p/security-audit}, \texttt{p/javascript}, \texttt{p/typescript}, and \texttt{p/owasp-top-ten} in JSON mode. CodeQL constructs a JavaScript database over the same workspace and analyzes it with the \texttt{javascript-security-and-quality} query suite (\texttt{codeql/javascript-queries}). ESLint uses the recommended rule set from \texttt{eslint-plugin-security} via a generated configuration file.

Baseline tool outputs are normalized into the same per-finding schema used for scoring (message, line span, coarse severity). Because tools report heterogeneous taxonomies, the harness infers the vulnerability \texttt{type} for baseline findings from rule identifiers, messages, and (when available) CWE metadata. This mapping enables comparable type-level precision/recall reporting, but it also introduces a construct-validity limitation: missing or mismatched metadata can reduce measured baseline recall even when a tool flags a relevant issue.

\subsection*{Comparability and Fairness Controls}

To support fair comparison across models and modes, the harness keeps the following controls fixed unless explicitly varied in the experiment:
\begin{itemize}
  \item identical dataset artifacts per run stage (vulnerable main set and secure overlay set),
  \item identical decoding/runtime limits (temperature, token budget, timeout, request delay),
  \item identical scoring logic for matching, parsing, and metric aggregation,
  \item identical execution order policy (sequential calls, no warm-up).
\end{itemize}

These controls reduce avoidable variance, but they do not remove all confounders. Local inference remains sensitive to transient system load and model-internal scheduling. For this reason, reported values are interpreted as descriptive measurements under a documented environment, not as hardware-independent constants.

\subsection*{Reproducibility Snapshot}

\begin{table}[H]
  \centering
  \caption{Run configuration used for reported results.}
  \label{tab:eval-reproducibility}
  \small
  \begin{tabular}{p{0.33\textwidth}p{0.6\textwidth}}
    \toprule
    Item & Value \\
    \midrule
    Dataset files & \texttt{all-test-cases.generated.json} (113 vulnerable) + \texttt{negatives-only.generated.json} (15 secure) \\
    Runs per sample & 3 (main ablation run) + 1 (negatives-only overlay) \\
    Prompt modes & LLM-only, LLM+RAG \\
    RAG settings & static security snippets, \texttt{k=5} \\
    Generation settings & \texttt{temperature=0.1}, \texttt{num\_predict=1000} \\
    Request controls & \texttt{timeoutMs=30000}, \texttt{requestDelayMs=500} \\
    Execution mode & sequential, no warm-up, no retries \\
    Models requested & \texttt{gemma3:1b}, \texttt{gemma3:4b}, \texttt{qwen3:4b}, \texttt{qwen3:8b}, \texttt{CodeLlama:latest} \\
    Model resolution & All models resolved to the same tags as requested in this run \\
    Software & Ollama 0.17.1, Node.js v20.19.5 \\
    Hardware/OS & Apple M4 Max (16 CPU cores, 64 GB RAM), macOS Darwin 25.3.0 (arm64) \\
    \bottomrule
  \end{tabular}
\end{table}

\subsection*{Artifact Provenance}

To document run provenance, Table~\ref{tab:eval-artifact-manifest} records the metadata fingerprint for the reported run.

\begin{table}[H]
  \centering
  \caption{Artifact provenance manifest for this thesis run.}
  \label{tab:eval-artifact-manifest}
  \footnotesize
  \begin{tabularx}{\textwidth}{p{0.34\textwidth}X}
    \toprule
    Field & Value \\
    \midrule
    Report repository commit & not captured by the harness in this run; repository HEAD at report build: \texttt{4fbeca1786ca} \\
    Run window (UTC) & 2026-02-27 08:36:19 to 2026-02-27 10:37:54 (main) + 2026-02-27 10:46:09 to 2026-02-27 10:51:51 (negatives overlay) \\
    Total runtime & 7\,295\,016 ms (main) + 341\,765 ms (negatives overlay) \\
    Execution policy & sequential order, no retries, no warm-up \\
    Dataset path & \path{code-guardian-extension/evaluation/datasets/all-test-cases.generated.json} + \path{code-guardian-extension/evaluation/datasets/negatives-only.generated.json} \\
    Invocation command & \texttt{node evaluation/evaluate-models.js --ablation --include-baselines --dataset=datasets/all-test-cases.generated.json --runs=3 --rag-k=5 --temperature=0.1 --num-predict=1000 --timeout-ms=30000 --delay-ms=500} \\
    Model fingerprint (\texttt{gemma3:1b}) & digest prefix \texttt{8648f39daa8f} \\
    Model fingerprint (\texttt{gemma3:4b}) & digest prefix \texttt{a2af6cc3eb7f} \\
    Model fingerprint (\texttt{qwen3:4b}) & digest prefix \texttt{359d7dd4bcda} \\
    Model fingerprint (\texttt{qwen3:8b}) & digest prefix \texttt{500a1f067a9f} \\
    Model fingerprint (\texttt{CodeLlama:latest}) & digest prefix \texttt{8fdf8f752f6e} \\
    \bottomrule
  \end{tabularx}
\end{table}

For archival reproducibility, the recommended artifact bundle includes: (i) raw JSON output from the harness, (ii) the exact run configuration object, (iii) model digests, and (iv) generated summary tables. These artifacts should be stored as supplementary material together with the thesis PDF.

\subsection{Run Configuration and Data Merging}
\label{sec:eval-run-merging}

This thesis evaluation uses a \textbf{two-run merged configuration} to balance statistical robustness on vulnerable samples with practical runtime constraints. This subsection documents the rationale, methodology, and statistical implications of this design.

\paragraph{Why two separate runs were necessary.}
The evaluation initially prioritized vulnerable-case coverage with repeated measurements (\texttt{runsPerSample=3}) to assess detection stability and reduce influence of transient model behavior. However, the original run configuration did not include sufficient secure-sample coverage for reliable false-positive rate (FPR) estimation—a critical usability metric for IDE-integrated tools.

Re-running the full ablation with both vulnerable and secure samples at \texttt{runsPerSample=3} would have required an additional 7+ hours of compute time. Instead, a pragmatic approach was adopted: execute a \textbf{negatives-only overlay run} (\texttt{runsPerSample=1}) to add 15 secure samples, then merge results for FPR computation while preserving the repeated vulnerable measurements for recall/precision.

\paragraph{How results were merged.}
The final thesis evaluation set combines:
\begin{itemize}
  \item \textbf{Vulnerable samples:} 113 samples $\times$ 3 runs = 339 evaluations (from \texttt{all-test-cases.generated.json})
  \item \textbf{Secure samples:} 15 samples $\times$ 1 run = 15 evaluations (from \texttt{negatives-only.generated.json})
\end{itemize}

Per-model metrics are computed as follows:
\begin{itemize}
  \item \textbf{Precision \& Recall:} Calculated over 339 vulnerable evaluations using type-level matching (expected vs predicted vulnerability labels).
  \item \textbf{False-Positive Rate (FPR):} Calculated over 15 secure evaluations using sample-level counting (a secure sample is FP if \emph{any} issue is emitted).
  \item \textbf{Latency:} Pooled across all 354 evaluations (339 vulnerable + 15 secure).
  \item \textbf{Parse Success Rate:} Pooled across all 354 evaluations.
\end{itemize}

\paragraph{Statistical implications.}
The merged design creates an \textbf{asymmetry in measurement robustness}: vulnerable-case metrics benefit from triple-sampling stability, while secure-case FPR is based on single-pass observations. This has several consequences:

\begin{enumerate}
  \item \textbf{Recall and precision estimates are more stable} due to repeated measurements reducing transient inference variance.
  \item \textbf{FPR estimates have wider confidence intervals} because the secure-sample count ($n=15$) is small and each sample is evaluated once. Wilson intervals (Table~\ref{tab:eval-ci-key}) reflect this uncertainty.
  \item \textbf{Mode-to-mode comparisons are descriptive only}: paired per-sample outputs were not exported by the harness, so differences between LLM-only and LLM+RAG are reported as run-level deltas, not inferential effects.
\end{enumerate}

\paragraph{Why pooled counts are still valid for descriptive reporting.}
Despite the asymmetry, the merged approach provides operationally useful evidence:
\begin{itemize}
  \item FPR observability is substantially improved (15 secure samples vs.\ 0--2 in early iterations).
  \item The primary evaluation goal is to \emph{compare model/mode profiles under controlled conditions}, not to establish statistically generalized population claims.
  \item Metrics are interpreted as \textbf{indicative measurements for this specific run configuration}, with explicit documentation of limitations (Section~\ref{sec:eval-summary}).
\end{itemize}

\paragraph{Future work: Unified repeated-run design.}
For production-scale evaluation, a unified design with matched \texttt{runsPerSample} for both vulnerable and secure sets would enable paired per-sample mode comparison and stronger statistical inference. The current harness can be extended to export per-sample mode-contingency outputs and support formal paired tests (e.g., McNemar's test for mode-to-mode detection differences).

\subsection*{Run-to-Run Variability (Descriptive)}

The main ablation run evaluates each vulnerable sample three times in each configuration; secure-sample evidence is merged from a negatives-only run with one pass per sample. Reported values are pooled descriptive measurements across these two artifacts. Because repeated runs reuse the same snippets and the final tables merge outputs from two runs, mode-to-mode differences are interpreted descriptively for this specific run.

\paragraph{Implications of merged-run reporting.}
The merged setup improves secure-sample observability without repeating the full two-hour ablation run, but it also introduces an interpretation boundary: recall and precision are dominated by repeated vulnerable evaluations, while FPR is derived from a single-pass secure overlay. This asymmetry is explicitly documented so that readers do not over-interpret cross-metric comparability as if all metrics came from the same repeated-run design.

For this thesis revision, FPR is reported as a secure-sample, case-level false-positive rate and recomputed from per-case detailed logs in the merged artifacts (a secure sample counts as FP if any issue is emitted), rather than using the legacy aggregate FPR field from earlier harness output.

\paragraph{Response parsing.}
The harness strips Markdown code fences if present and then attempts to parse the remaining content as JSON. Responses that fail to parse are counted toward the parse failure rate and are scored as producing no issues, which impacts recall on vulnerable samples.

\subsection*{Running the Evaluation}

The evaluation can be reproduced by running:
\begin{lstlisting}[language=Java, caption={Running the Code Guardian evaluation harness}, label={lst:eval-run}]
cd code-guardian-extension
node evaluation/evaluate-models.js --ablation --include-baselines \
  --dataset=datasets/all-test-cases.generated.json --runs=3 --rag-k=5 --temperature=0.1 \
  --num-predict=1000 --timeout-ms=30000 --delay-ms=500
node evaluation/evaluate-models.js --ablation --include-baselines \
  --dataset=datasets/negatives-only.generated.json --runs=1 --rag-k=5 --temperature=0.1 \
  --num-predict=1000 --timeout-ms=30000 --delay-ms=500
\end{lstlisting}

The script produces per-model metrics and can be extended to emit JSON/Markdown reports under \path{code-guardian-extension/evaluation/logs/} for inclusion in the thesis appendix. For strict comparability with the reported tables, the evaluated model list should match the run configuration (\texttt{gemma3:1b}, \texttt{gemma3:4b}, \texttt{qwen3:4b}, \texttt{qwen3:8b}, and \texttt{CodeLlama:latest}).

The reported results in this thesis combine a \texttt{runsPerSample=3} ablation run (vulnerable set) with a \texttt{runsPerSample=1} negatives-only overlay to recover secure-sample coverage for FPR computation.
