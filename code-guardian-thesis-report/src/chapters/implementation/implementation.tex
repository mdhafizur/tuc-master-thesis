Building on the conceptual architecture introduced in the previous chapter, this chapter describes the implementation of \textit{Code Guardian}: a privacy-preserving vulnerability detection and repair assistant integrated into Visual Studio Code. It details the technology stack, end-to-end workflow, core component implementation, orchestration modes (LLM-only vs.\ LLM+RAG), and the user interface exposed inside the IDE.

The implementation focus is intentionally engineering-oriented: how design requirements are translated into runtime behavior, guardrails, and failure handling on real developer hardware. Where relevant, this chapter emphasizes reproducibility-relevant details (execution order, schema contracts, caching behavior, and local privacy boundaries) so that evaluation outcomes in Chapter~\ref{chap:evaluation} can be interpreted against concrete implementation mechanisms rather than high-level architecture alone.

\section{Technology Stack and Rationale}
\label{sec:tech-stack}

Implementation of the \textit{Invox} system follows a modular monolith architecture that balances development simplicity with operational flexibility. A single Node.js application hosts all five agents, while specialized services—such as vector search, embedding generation, and external LLMs—operate externally. This hybrid approach preserves modular boundaries without incurring the operational overhead typical of fully distributed microservices \cite{fowler2015monolith,microservices_newman}.

\subsection*{Architecture Pattern: Modular Monolith with External Services}

Invox adopts a modular monolith pattern in which the STT, RAG, IE, CF, and VER agents are implemented as cohesive modules within one Node.js service. This yields a unified codebase, consistent versioning, simplified debugging, and low-latency in-process communication. Clear module interfaces maintain separation of concerns despite co-location.

External components are used where the benefits are decisive. \textbf{OpenSearch} provides vector similarity search; a dedicated embedding service generates domain-relevant vectors; and LLM providers (OpenAI, Google) perform model inference. This division allows the core pipeline to remain lightweight while delegating specialized computation to optimized services.

\subsection*{Containerization and Deployment}

Deployment relies on \textbf{Docker} to ensure reproducibility, isolation, and consistent behavior across environments. Containers encapsulate dependencies, simplify distribution, and enable horizontal scaling by replicating services. \textbf{Docker Compose} orchestrates the multi-service environment, coordinating the main application alongside OpenSearch and auxiliary components.

\subsection*{Backend Implementation}

Backend development is carried out in \textbf{Node.js} with \textbf{Express}, providing an adaptable foundation for the multi-agent workflow. LLM invocation is abstracted through the \textbf{Vercel AI SDK}, which enables a plugin-like architecture for switching between OpenAI, Google, Anthropic, and future providers with minimal changes. Type-safe communication between backend and frontend is achieved using \textbf{tRPC}, with \textbf{JSON} as the shared data format for inter-agent communication.

\subsection*{External Service Integration}

\textbf{OpenSearch} acts as the vector index for the RAG agent, storing dense embeddings and enabling efficient retrieval of semantically related examples. Vector indexing significantly accelerates similarity search and supports context-rich extraction. Embeddings are generated using the \texttt{intfloat/multilingual-e5-large-instruct} model (1024 dimensions), selected for its strong multilingual performance and favorable trade-off between accuracy and computational cost. This model provides robust semantic representations for diverse input conditions.

\subsection*{Frontend Interface}

The user interface is built in \textbf{React}, chosen for its wide adoption, strong ecosystem, and suitability for building interactive components that match the system’s modular structure. React's predictable rendering model ensures responsive interactions when reviewing extracted fields and modifying templates. 

\subsection*{Data Persistence and Retrieval}

Persistent storage is managed by \textbf{PostgreSQL}, which offers strong consistency, reliability, and efficient querying for structured data. Finalized templates are stored relationally, while OpenSearch serves as the retrieval layer for RAG by indexing dense vector embeddings. This combination provides both robust persistence and high-quality semantic search.

\subsection*{Authentication and Security}

Authentication and authorization are handled by \textbf{Keycloak}, which supports OAuth2 and OpenID Connect \cite{rfc6749}. Role-based policies ensure that only authorized users can submit, review, or modify templates, supporting secure multi-user workflows.

\subsection*{Architecture Rationale}

A modular monolith provides a pragmatic balance between maintainability, performance, and future extensibility. It avoids the operational complexity of microservices while retaining clear module boundaries. The plugin-based LLM integration ensures adaptability as new models become available, and containerized deployment guarantees reproducibility and reliability across environments. Together, these choices provide a scalable and future-proof foundation for \textit{Invox}.

\section{System Workflow}
\label{sec:system-workflow}

This section describes the operational workflow of the Invox system, tracing the transformation of unstructured input into verified structured templates. The workflow implements the conceptual pipeline from Chapter~\ref{chap:concept} through a sequence of processing stages with user-driven quality control.

\subsection*{End-to-End Processing Flow}

The end-to-end workflow of Invox’s five-agent pipeline is illustrated in Figure~\ref{fig:agent-pipeline}. Starting from user input, the system proceeds through six sequential stages:

\begin{enumerate}
    \item \textbf{Input Reception}: Users provide input through speech (audio recording) or direct text entry via the web interface. The system accepts both modalities and routes them appropriately through the processing pipeline.
    
    \item \textbf{Speech Recognition}: For audio inputs, the system invokes the Whisper ASR service to generate transcripts with confidence scores and timestamps. Text inputs proceed directly to the next stage.
    
    \item \textbf{Context Retrieval}: The RAG agent queries the vector database to retrieve the most relevant historical templates and examples, providing contextual guidance for the extraction process.
    
    \item \textbf{Information Extraction}: Based on the selected strategy (S1–S4), the system processes the input:
    \begin{itemize}
        \item \textbf{S1 (Single-Pass Full Input)}: A single language model extracts all template fields in one pass
        \item \textbf{S2 (Iterative Single-Field Processing)}: Multiple LLM calls extract each field independently  
        \item \textbf{S3/S4 (Multi-LLM Consensus (Full), Multi-LLM Consensus (Iterative))}: Multiple models generate candidate answers, with a dedicated judge LLM selecting the optimal response through comparative analysis
    \end{itemize}
    
    \item \textbf{Consistency Enforcement}: The CF agent normalizes dates to ISO format, standardizes entity names, and enforces schema compliance through deterministic transformation rules.
    
    \item \textbf{Quality Assessment and Presentation}: The system assigns confidence scores to each extracted field and presents the complete template to the user for review and potential correction.
\end{enumerate}

\subsection*{User-Driven Quality Control}

The system employs a user-in-the-loop approach where quality assurance is primarily driven by human judgment:

\textbf{Confidence-Based Presentation}: Fields with lower confidence scores are visually highlighted, directing user attention to potentially problematic extractions.

\textbf{Multi-Model Decision Making}: In strategies S3 and S4, the judge LLM evaluates candidate answers from multiple models, selecting the most appropriate response based on coherence, accuracy, and alignment with the input context.

\textbf{User-Initiated Reprocessing}: If users are unsatisfied with results, they can modify their input and resubmit for reprocessing, creating an iterative refinement cycle driven by human assessment rather than automated detection.

\subsection*{Knowledge Accumulation}

Successfully processed templates are indexed in the vector database, enabling the RAG agent to leverage an expanding knowledge base for improved context retrieval in future processing cycles.
\section{Core Component Implementation}
\label{sec:impl-components}

This section details the concrete implementation of Code Guardian’s main components. The prototype is implemented as a single VS Code extension located in \texttt{code-guardian/}. Its primary tasks are (i) extracting relevant code context inside the IDE, (ii) invoking local LLM inference, (iii) optionally grounding the prompt with retrieved security knowledge (RAG), and (iv) presenting findings and repairs in an IDE-native way.

\subsection{Extension Entry Point and Event Wiring}
\label{subsec:impl-extension}

The extension entry point registers event listeners and commands. Real-time analysis is triggered on document changes and is debounced (default: 800\,ms) to avoid excessive inference calls during typing. On-demand commands support analyzing a selection, a full file, or scanning the workspace. Findings are reported through the VS Code diagnostics API so they appear inline and in the Problems panel.

\subsection{Context Extraction and Scoping}
\label{subsec:impl-context}

Code Guardian extracts the smallest useful unit of code for interactive analysis: the enclosing function at the cursor position. This design reduces prompt size and improves responsiveness without requiring full-program analysis. Function extraction is implemented as a lightweight syntactic pass (\texttt{code-guardian/src/functionExtractor.ts}) and returns both the extracted snippet and its start-line offset within the original document. When a snippet is analyzed, the offset is later used to map model-reported line numbers back to the full file, so that diagnostics are placed correctly in the editor.

For explicit commands, the scope expands to (i) a selected region (or current line if no selection exists) and (ii) the full file. These scopes prioritize completeness and are typically used for deeper inspection or before committing changes.

\subsection{LLM Analyzer (Local JSON-Only Output)}
\label{subsec:impl-analyzer}

The analyzer performs local inference through Ollama and requires the model to return \emph{only} a JSON array of findings. The output schema is intentionally minimal to keep parsing robust and IDE rendering straightforward.

\begin{lstlisting}[language=TypeScript, caption={Security issue output schema used by Code Guardian}, label={lst:cg-issue-schema}]
interface SecurityIssue {
  message: string;
  startLine: number;   // 1-based
  endLine: number;     // 1-based
  suggestedFix?: string;
}
\end{lstlisting}

To increase reliability, the implementation includes:
\begin{itemize}
  \item \textbf{Schema-constrained prompting:} the system prompt instructs strict JSON-only output.
  \item \textbf{Defensive parsing:} Markdown fences and stray quotes are removed, and the first JSON array substring is extracted when needed.
  \item \textbf{Retry logic:} transient errors (timeouts, temporary unavailability) are retried with exponential backoff.
\end{itemize}

\paragraph{Failure handling and safe defaults.}
In a developer tool, a false crash is often worse than a missed warning because it interrupts the workflow. Therefore, non-recoverable failures (e.g., repeated timeouts, model-not-found) are handled by showing a user-facing message and returning an empty issue list. This keeps the editor responsive while preserving explicit control over model configuration (e.g., prompting the user to pull a missing model).

\subsection{Diagnostics Mapping and Localization}
\label{subsec:impl-localization}

The diagnostic adapter converts the model’s 1-based line numbering into VS Code’s 0-based ranges and clamps indices to valid document bounds (\texttt{code-guardian/src/diagnostic.ts}). When a function snippet is analyzed, the previously computed start-line offset is added to each finding’s range. If a suggested fix is included, it is attached to the diagnostic as related information and surfaced as a quick-fix action.

\subsection{RAG Manager (Local Retrieval)}
\label{subsec:impl-rag}

When enabled, the RAG manager maintains a local security knowledge base and a persistent vector index. Knowledge items are embedded using a local embedding model accessed through Ollama (\texttt{nomic-embed-text}) and stored in an HNSW vector store. For each analysis request, the manager retrieves the top-$k$ relevant snippets and augments the prompt with (i) short vulnerability definitions and (ii) mitigation guidance. This grounding supports consistency and explainability (R1--R3).

\paragraph{Indexing and chunking.}
Knowledge entries are chunked using a recursive text splitter to balance semantic coherence and retrieval recall. The index is persisted under the extension’s storage path so it can be reused across sessions without rebuilding. RAG initialization is performed lazily at runtime to avoid slowing down extension activation when retrieval is not used.

\subsection{Vulnerability Knowledge Updates}
\label{subsec:impl-knowledge}

To keep retrieved content current, the vulnerability data manager periodically refreshes public security sources (e.g., OWASP Top~10 entries, a curated set of CWE patterns, and a configurable number of recent CVEs). Retrieved metadata is cached on disk and only public vulnerability information is fetched. No user source code or extracted code context is transmitted off-device, preserving the privacy goal (R5).

\paragraph{Offline fallback.}
Because network access may be restricted in sensitive environments, the system includes a minimal baseline knowledge bundle that is used when updates fail. This ensures that RAG-enabled prompting remains functional offline, even though coverage is reduced relative to a refreshed knowledge base.

\subsection{Diagnostics and Quick Fixes}
\label{subsec:impl-diagnostics}

Findings are mapped to VS Code diagnostics with appropriate text ranges. If the model provides a \texttt{suggestedFix}, a quick-fix code action is offered. Fixes are never applied automatically: the developer must confirm application, which maintains human control and reduces the risk of unintended behavioral changes (R4).

\subsection{Caching and Responsiveness Mechanisms}
\label{subsec:impl-cache}

To avoid repeated inference on unchanged snippets, Code Guardian caches analysis results keyed by the code snippet, active model, and prompt mode (RAG enabled/disabled) (\texttt{code-guardian/src/analysisCache.ts}). In addition to caching, real-time analysis is debounced to reduce request volume during rapid edits. Together, these mechanisms reduce redundant local LLM calls and make continuous feedback feasible on developer hardware (R6), while also improving run-to-run consistency by limiting stochastic re-analysis.

\subsection{Workspace Scanner and Dashboard}
\label{subsec:impl-workspace}

For project-level visibility, Code Guardian includes a workspace scanner that analyzes JavaScript and TypeScript files in batch and aggregates results into a dashboard WebView. The dashboard summarizes severity distribution and surfaces the most vulnerable files, supporting risk-based prioritization and iterative hardening.

The scanner enumerates files by extension, excludes dependency folders (e.g., \texttt{node\_modules}), and skips very large files to bound runtime. Because the JSON-only analyzer does not mandate a severity field, the scanner applies a conservative keyword-based severity heuristic to support coarse prioritization in the dashboard; this is treated as a presentation aid rather than a ground-truth classifier.

\subsection{Privacy Boundary and Threat Model Considerations}
\label{subsec:impl-privacy-boundary}

Code Guardian’s privacy boundary is defined at the point of inference: analyzed code and any extracted context are sent only to the local Ollama server. The system does not transmit source code to external services. When knowledge updates are enabled, only public vulnerability metadata is fetched; this data is cached locally and then used to ground prompts.

From a threat-model perspective, the main risks are \emph{prompt injection} (attacker-controlled comments or strings that attempt to override the analysis instruction) and \emph{retrieval poisoning} (malicious or misleading knowledge entries). The current prototype mitigates these risks primarily through strict output contracts and by keeping retrieval sources scoped to curated security data; Chapter~\ref{chap:future-work} outlines stronger mitigations such as provenance tracking, allowlisting, and retrieval sanitization.

\section{Strategy Implementation}
\label{sec:impl-strategies}

This section presents the practical realization of the four architectural strategies introduced in Section~\ref{sec:architectural-strategies}. While the conceptual motivations were discussed earlier, the focus here is on their concrete execution: preparation of transcripts and few-shot examples, invocation of language models, and consolidation of outputs. Simplified TypeScript sketches of the core procedures are included in Appendix~\ref{app:strategy-impl}.

\subsection{S1 Implementation: Single-Pass Full-Input}
\label{subsec:impl-s1}

The S1 strategy is implemented through the \texttt{singleLlmAllField} function, which processes the entire template in a single model call. The system first constructs a combined transcript and retrieves a small set of examples from the RAG agent. All fields are represented within a unified schema, and the prompt incorporates task instructions, field definitions, transcripts, current values, and retrieved examples.

A single \texttt{generateObject} call produces a complete candidate template. The output is subsequently normalized via the CF agent’s rules and verified once in aggregate. The strategy is therefore characterized by a single prompt, a single model invocation, and batch-style post-processing. While efficient and simple to orchestrate, S1 provides limited error isolation, as malformed responses may affect all fields simultaneously.

\subsection{S2 Implementation: Iterative Single-Field}
\label{subsec:impl-s2}

S2, implemented by \texttt{singleLlmOneField}, introduces field-level decomposition. After preparing the combined transcript and retrieving few-shot examples, the system constructs individual prompts for each field, containing only the information relevant to that field, including filtered examples.

Each field is processed by an independent asynchronous task. Outputs pass through normalization and are merged into the final template. Errors are isolated to the corresponding field; upon failure, the system reverts to a sensible fallback (typically the previously known value). S2 thus provides improved robustness relative to S1 through parallelized extraction, field-specific prompting, and localized error handling.

\subsection{S3 Implementation: Multi-LLM Consensus}
\label{subsec:impl-s3}

The S3 strategy is implemented using \texttt{dualLlmAllField}, which performs model-level parallelism. A unified prompt and schema are constructed as in S1, but two model providers (e.g., GPT-4 and Gemini~2.0 Flash) are invoked in parallel. Because both models receive identical prompts, differences in their outputs reflect model behaviour, not prompt variation.

The two candidates are forwarded to an ensemble verifier, which applies rule-based checks and a judge LLM to select, for each field, the more reliable value or retain the original value when neither candidate is adequate. A finalized template and confidence metadata are produced. S3 is therefore defined by parallel model inference, an explicit consensus phase, and structured documentation of ensemble decisions.

\subsection{S4 Implementation: Multi-LLM Per-Field}
\label{subsec:impl-s4}

S4, implemented through \texttt{multiLlmOneField}, combines field-wise decomposition with multi-model consensus. For each field, the system constructs a tailored prompt and invokes both model providers. Their outputs are passed to a per-field verifier, which applies the same decision logic as in S3 but at field granularity.

Fields are processed sequentially to avoid rate-limit interactions and to simplify auditing. Each iteration returns a \texttt{FilledField} entry after consensus and normalization. This yields the highest robustness and interpretability among the strategies, though at the cost of substantially increased computational overhead.

\subsection{Strategy Orchestration Infrastructure}
\label{subsec:impl-orchestration-strategies}

All strategies are executed through a dedicated \texttt{StrategyOrchestrator}, which exposes a uniform entry point (\texttt{executeStrategy}). The orchestrator maps the selected identifier (S1--S4) to a concrete implementation adhering to a shared \texttt{StrategyImplementation} interface. 

Common infrastructure ensures consistent error handling, telemetry, and configuration. Strategy-specific parameters (model names, temperatures, provider settings) are injected through environment-based configuration, and all outputs are serialized into a unified \texttt{FinalTemplate} structure, ensuring that downstream evaluation and analytics remain strategy-agnostic. A minimal TypeScript sketch of the orchestrator is provided in Appendix~\ref{app:strategy-impl}.

\section{User Interface}
\label{sec:user-interface}

The user interface implements the human-in-the-loop quality control paradigm established in Section~\ref{sec:system-workflow}. The interface design focuses on efficient template filling through voice input and interactive correction workflows.

\subsection{Authentication and Template Access}
\label{subsec:ui-authentication}

\begin{figure}[H]
  \centering
  \includegraphics[width=1.0\linewidth]{images/login_interface.png}
  \caption{INVOX authentication interface with Keycloak integration}
  \label{fig:login-interface}
\end{figure}

The system implements role-based access through a centralized authentication interface (Figure~\ref{fig:login-interface}). Users authenticate via Keycloak-managed credentials, which determine which template types they can access. Security personnel see incident reporting templates, while medical staff access patient handover forms. Users cannot create new template schemas through the interface—they can only fill templates assigned to their role.

\subsection{Template Filling Interface}
\label{subsec:ui-template-filling}

\begin{sidewaysfigure}
  \centering
  \includegraphics[width=1.0\linewidth]{images/template_interface.png}
  \caption{Template-filling interface used in Invox. The left panel offers voice recording and chat interaction for describing an incident, while the right panel displays the structured form populated by the pipeline. The layout illustrates how transcription, automated extraction, and user review occur within a unified workflow.}
  \label{fig:template-interface}
\end{sidewaysfigure}

The interface consists of three main components (Figure~\ref{fig:template-interface}): the audio recording panel at the bottom-left, the structured template form on the right, and an interactive chat panel for clarifications.

\subsubsection{Audio Recording Panel}

The recording panel provides a "Start" button to initiate audio capture. During recording, the button changes to "Stop \& Process." When clicked, the system transcribes the audio and extracts template values using the selected strategy. A strategy selector menu allows users to choose between S1, S2, S3, or S4 before processing, enabling them to balance speed versus accuracy based on the template's importance.

\subsubsection{Structured Template Form}

The right side shows the structured template, with fields grouped by category (Incident Type, Perpetrator, Target, Victim, Weapon). As audio is processed, the system auto-populates the fields using extraction results. Each field uses an appropriate widget—dropdowns for enumerations, text areas for descriptions, and dedicated inputs for dates or numeric values. Confidence indicators mark AI-filled entries, and low-confidence fields are highlighted for review.

Users may edit values at any time, lock fields to protect them during later recordings, and inspect whether a value was extracted or manually entered. Before submission, the form validates all entries against the template schema.

\subsubsection{Interactive Chat Panel}

The chat panel provides real-time feedback during template filling. When the system identifies missing required fields, conflicting information, or ambiguous extractions, it posts messages requesting clarification. For example, if the perpetrator field contains uncertain information, the chat prompts the user to provide more specific details. This interactive guidance reduces iteration cycles by identifying issues immediately rather than after submission.

\subsection{Template Export}

Once users complete template review, the interface provides export options in JSON, Excel, and PDF formats through a download button group. All templates are automatically persisted to PostgreSQL upon processing, ensuring data preservation regardless of export actions. Having established the complete implementation of the Invox system—from backend agent architecture to frontend user interface—the next chapter evaluates this implementation empirically. Chapter~\ref{chap:evaluation} applies the system to the MUC-4 benchmark and measures its performance against the six requirements (R1–R6) established in Chapter~\ref{sec:requirements}. The evaluation compares the four architectural strategies quantitatively, examining their trade-offs in accuracy, cost, latency, and user satisfaction.
