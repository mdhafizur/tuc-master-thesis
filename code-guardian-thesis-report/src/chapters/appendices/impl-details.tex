\section{Prompt Contract (JSON-Only Output)}
\label{app:prompt-contract}

Code Guardian relies on a strict output contract so findings can be converted into IDE diagnostics reliably. The analyzer instructs the local model to return \emph{only} a JSON array of security issues.

\begin{lstlisting}[language=Java, caption={JSON-only response schema (conceptual)}, label={lst:appendix-json-schema}]
[
  {
    "message": "Issue description",
    "startLine": 1,
    "endLine": 3,
    "suggestedFix": "Optional secure alternative"
  }
]
\end{lstlisting}

\subsection*{Robustness in practice}

Even with explicit instructions, some models occasionally emit Markdown fences or additional text. The prototype therefore applies defensive parsing: it removes common code-block markers and extracts the first JSON array substring before attempting to parse. This is a pragmatic mechanism to improve structured-output robustness for IDE integration.

\section{Runtime Guardrails}
\label{app:guardrails}

To keep the extension usable on developer hardware (R6), Code Guardian includes conservative guardrails:
\begin{itemize}
  \item \textbf{Debounce interval:} 800\,ms for real-time analysis after document changes.
  \item \textbf{Function scope size limit:} extracted functions larger than 2000 characters are skipped in real-time mode.
  \item \textbf{File scope size limit:} full-file analysis is skipped above 20{,}000 characters.
  \item \textbf{Workspace scan file size limit:} files larger than 500\,KB are skipped in batch scans.
\end{itemize}

In addition, analysis results are cached using an LRU-style cache (100 entries, 30-minute TTL) to reduce redundant inference calls during iterative edits.

\section{Key Configuration Options}
\label{app:config}

The extension exposes user configuration through VS Code settings. The most relevant options are:
\begin{itemize}
  \item \texttt{codeGuardian.model}: default Ollama model for analysis
  \item \texttt{codeGuardian.ollamaHost}: Ollama base URL (default: \url{http://localhost:11434})
  \item \texttt{codeGuardian.enableRAG}: enable/disable retrieval augmentation
\end{itemize}

\section{VS Code Commands (Prototype)}
\label{app:commands}

The prototype exposes the following main commands via the Command Palette:
\begin{itemize}
  \item \textbf{Analyze selected code with AI}: opens the interactive analysis view for a selection or current line.
  \item \textbf{Analyze full file}: runs the structured diagnostics pipeline over the active document.
  \item \textbf{Contextual Q\&A}: opens a WebView for asking security questions with user-selected file/folder context.
  \item \textbf{Select AI model}: lists available local Ollama models and switches the active model.
  \item \textbf{Manage RAG knowledge base}: view/add/search knowledge, rebuild the vector store, and update vulnerability data.
  \item \textbf{Toggle RAG}: enables/disables retrieval augmentation through settings.
  \item \textbf{Update vulnerability data}: refreshes cached public metadata used for the knowledge base.
  \item \textbf{View cache statistics}: inspects the analysis cache and supports clearing/resetting statistics.
  \item \textbf{Workspace security dashboard}: performs a batch scan and displays an aggregate dashboard.
\end{itemize}

\section{Evaluation Harness Location}
\label{app:eval-harness}

The evaluation script used in Chapter~\ref{chap:evaluation} is located at \path{code-guardian-extension/evaluation/evaluate-models.js}. The curated datasets are located in \path{code-guardian-extension/evaluation/datasets/}.
